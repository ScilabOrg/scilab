% Scilab ( http://www.scilab.org/ ) - This file is part of Scilab
% Copyright (C) 2008-2010 - Digiteo - Michael Baudin
%
% This file must be used under the terms of the CeCILL.
% This source file is licensed as described in the file COPYING, which
% you should have received as part of this distribution.  The terms
% are also available at
% http://www.cecill.info/licences/Licence_CeCILL_V2-en.txt

\section{The Pythagorean sum}

In this section, we analyze the computation of the Pythagorean sum,
which is used in two different computations, that is the norm of a complex
number and the 2-norm of a vector of real values.

In the first part, we briefly present the mathematical formulas for these 
two computations.
We then present the naive algorithm based on these mathematical formulas. 
In the second part, we make some experiments in Scilab and compare our
naive algorithm with the \scifun{abs} and \scifun{norm} Scilab functions.
In the third part, we analyze 
why and how floating point numbers must be taken into account when the 
Pythagorean sum is to compute.

\subsection{Theory}

\subsection{Experiments}

% TODO : compare both abs and norm.

\lstset{language=scilabscript}
\begin{lstlisting}
// Straightforward implementation
function mn2 = mynorm2(a,b)
  mn2 = sqrt(a^2+b^2)
endfunction
// With scaling
function mn2 = mypythag1(a,b)
  if (a==0.0) then
    mn2 = abs(b);
  elseif (b==0.0) then
    mn2 = abs(a);
  else
    if (abs(b)>abs(a)) then
      r = a/b;
      t = abs(b);
    else
      r = b/a;
      t = abs(a);
    end
    mn2 = t * sqrt(1 + r^2);
  end
endfunction
// With Moler & Morrison's
// At most 7 iterations are required.
function mn2 = mypythag2(a,b)
  p = max(abs(a),abs(b))
  q = min(abs(a),abs(b))
  //index = 0
  while (q<>0.0)
    //index = index + 1
    //mprintf("index = %d, p = %e, q = %e\n",index,p,q)
    r = (q/p)^2
    s = r/(4+r)
    p = p + 2*s*p
    q = s * q
  end
  mn2 = p
endfunction
function compare(x)
  mprintf("Re(x)=%e, Im(x)=%e\n",real(x),imag(x));
  p = abs(x);
  mprintf("%20s : %e\n","Scilab",p);
  p = mynorm2(real(x),imag(x));
  mprintf("%20s : %e\n","Naive",p);
  p = mypythag1(real(x),imag(x));
  mprintf("%20s : %e\n","Scaling",p);
  p = mypythag2(real(x),imag(x));
  mprintf("%20s : %e\n","Moler & Morrison",p);
endfunction
// Test #1 : all is fine
x = 1 + 1 * %i;
compare(x);
// Test #2 : more difficult when x is large
x = 1.e200 + 1 * %i;
compare(x);
// Test #3 : more difficult when x is small
x = 1.e-200 + 1.e-200 * %i;
compare(x);
\end{lstlisting}

\begin{lstlisting}
***************************************
Example #1 : simple computation with Scilab 5.1
x(1)=1.000000e+000, x(2)=1.000000e+000
              Scilab : 1.414214e+000
               Naive : 1.414214e+000
             Scaling : 1.414214e+000
    Moler & Morrison : 1.414214e+000
***************************************
Example #2 : with large numbers ?
              Scilab : Inf
               Naive : Inf
             Scaling : 1.000000e+200
    Moler & Morrison : 1.000000e+200
***************************************
Example #3 : with small numbers ?
x(1)=1.000000e-200, x(2)=1.000000e-200
              Scilab : 0.000000e+000
               Naive : 0.000000e+000
             Scaling : 1.414214e-200
    Moler & Morrison : 1.414214e-200
***************************************
> Conclusion : Scilab is naive !
Octave 3.0.3
***************************************
octave-3.0.3.exe:29> compare(x);
***************************************
x(1)=1.000000e+000, x(2)=1.000000e+000
              Octave : 1.414214e+000
               Naive : 1.414214e+000
             Scaling : 1.414214e+000
    Moler & Morrison : 1.414214e+000
***************************************
x(1)=1.000000e+200, x(2)=1.000000e+000
              Octave : 1.000000e+200
               Naive : Inf
             Scaling : 1.000000e+200
    Moler & Morrison : 1.000000e+200
***************************************
octave-3.0.3.exe:33> compare(x)
x(1)=1.000000e-200, x(2)=1.000000e-200
              Octave : 1.414214e-200
               Naive : 0.000000e+000
             Scaling : 1.414214e-200
    Moler & Morrison : 1.414214e-200
***************************************
> Conclusion : Octave is not naive !

With complex numbers.
***************************************

Re(x)=1.000000e+000, Im(x)=1.000000e+000
              Scilab : 1.414214e+000
               Naive : 1.414214e+000
             Scaling : 1.414214e+000
    Moler & Morrison : 1.414214e+000
***************************************
Re(x)=1.000000e+200, Im(x)=1.000000e+000
              Scilab : 1.000000e+200
               Naive : Inf
             Scaling : 1.000000e+200
    Moler & Morrison : 1.000000e+200
***************************************
Re(x)=1.000000e-200, Im(x)=1.000000e-200
              Scilab : 1.414214e-200
               Naive : 0.000000e+000
             Scaling : 1.414214e-200
    Moler & Morrison : 1.414214e-200
***************************************
> Conclusion : Scilab is not naive !
\end{lstlisting}

\subsection{Explanations}

\subsection{References}

The paper by Moler and Morrisson 1983 \cite{journals/ibmrd/MolerM83} gives an 
algorithm to compute the Pythagorean sum $a\oplus b = \sqrt{a^2 + b^2}$
without computing their squares or their square roots. Their algorithm is based on a cubically
convergent sequence.
The BLAS linear algebra suite of routines \cite{900236} includes the SNRM2, DNRM2
and SCNRM2 routines which compute the euclidian norm of a vector.
These routines are based on Blue \cite{355771} and Cody \cite{Cody:1971:SEF}.
In his 1978 paper \cite{355771}, James Blue gives an algorithm to compute the 
Euclidian norm of a n-vector $\|x\| = \sqrt{\sum_{i=1,n}x_i^2}$. 
The exceptional values of the \scifun{hypot} operator are defined as the 
Pythagorean sum in the IEEE 754 standard \cite{P754:2008:ISF,ieee754-1985}.
The \scifun{\_\_ieee754\_hypot(x,y)} C function is implemented in the 
Fdlibm software library \cite{fdlibm} developed by Sun Microsystems and 
available at netlib. This library is used by Matlab \cite{matlab-hypot}
and its \scifun{hypot} command.




