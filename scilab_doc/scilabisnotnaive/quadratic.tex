\section{Quadratic equation}

In this section, we detail the computation of the roots of a quadratic polynomial.
As we shall see, there is a whole world from the mathematics formulas to the 
implementation of such computations. In the first part, we briefly report the formulas which allow to 
compute the real roots of a quadratic equation with real coefficients.
We then present the na�ve algorithm based on these mathematical formulas. 
In the second part, we make some experiments in Scilab and compare our
na�ve algorithm with the \emph{roots} Scilab primitive.
In the third part, we analyse 
why and how floating point numbers must be taken into account when the 
implementation of such roots is required.

\subsection{Theory}

We consider the following quadratic equation, with real 
coefficients $a, b, c \in \RR$ \cite{wikipediaquadratic,wikipedialossofsign,mathworldquadratic} :

\begin{eqnarray}
a x^2 + b x + c = 0.
\end{eqnarray}

The real roots of the quadratic equations are
\begin{eqnarray}
x_- &=& \frac{-b- \sqrt{b^2-4ac}}{2a}, \label{real:x-}\\
x_+ &=& \frac{-b+ \sqrt{b^2-4ac}}{2a}, \label{real:x+}
\end{eqnarray}
with the hypothesis that the discriminant $\Delta=b^2-4ac$
is positive.

The naive, simplified, algorithm which computes the roots of the 
quadratic is presented in figure \ref{naive-quadratic}.

\begin{figure}[htbp]
\begin{algorithmic}
\STATE $\Delta\gets b^2-4ac$
\STATE $s\gets \sqrt{\Delta}$
\STATE $x_-\gets (-b-s)/(2a)$
\STATE $x_+\gets (-b+s)/(2a)$
\end{algorithmic}
\caption{Naive algorithm to compute the real roots of a quadratic equation}
\label{naive-quadratic}
\end{figure}

\subsection{Experiments}

The following Scilab function is a straitforward implementation
of the previous formulas.

\lstset{language=Scilab}
\lstset{numbers=left}
\lstset{basicstyle=\footnotesize}
\lstset{keywordstyle=\bfseries}
\begin{lstlisting}
function r=myroots(p)
  c=coeff(p,0);
  b=coeff(p,1);
  a=coeff(p,2);
  r=zeros(2,1);
  r(1)=(-b+sqrt(b^2-4*a*c))/(2*a);
  r(2)=(-b-sqrt(b^2-4*a*c))/(2*a);
endfunction
\end{lstlisting}

The goal of this section is to show that some additionnal
work is necessary to compute the roots of the quadratic equation
with sufficient accuracy.
We will especially pay attention to rounding errors and 
overflow problems.
In this section, we show that the \emph{roots} command 
of the Scilab language is not \emph{naive}, in the sense that it 
takes into account for the floating point implementation details 
that we will see in the next section.



\subsubsection{Rounding errors}

We analyse the rounding errors which are 
appearing when the discriminant of the quadratic equation 
is such that $b^2\approx 4ac$.
We consider the following quadratic equation 
\begin{eqnarray}
\epsilon x^2 + (1/\epsilon)x - \epsilon = 0
\end{eqnarray}
with $\epsilon=0.0001=10^{-4}$.

The two real solutions of the quadratic equation are
\begin{eqnarray}
x_- &=& \frac{-1/\epsilon- \sqrt{1/\epsilon^2+4\epsilon^2}}{2\epsilon} \approx  -1/\epsilon^2, \\
x_+ &=& \frac{-1/\epsilon+ \sqrt{1/\epsilon^2+4\epsilon^2}}{2\epsilon} \approx  \epsilon^2
\end{eqnarray}

The following Scilab script shows an example of the computation
of the roots of such a polynomial with the \emph{roots}
primitive and with a naive implementation.
Only the positive root $x_+ \approx \epsilon^2$ is considered in this 
test (the $x_-$ root is so that $x_- \rightarrow -\infty$ in both 
implementations).

\lstset{language=Scilab}
\lstset{numbers=left}
\lstset{basicstyle=\footnotesize}
\lstset{keywordstyle=\bfseries}
\begin{lstlisting}
p=poly([-0.0001 10000.0 0.0001],"x","coeff");
e1 = 1e-8;
roots1 = myroots(p);
r1 = roots1(1);
roots2 = roots(p);
r2 = roots2(1);
error1 = abs(r1-e1)/e1;
error2 = abs(r2-e1)/e1;
printf("Expected : %e\n", e1);
printf("Naive method : %e (error=%e)\n", r1,error1);
printf("Scilab method : %e (error=%e)\n", r2, error2);
\end{lstlisting}

The script then prints out :

\begin{verbatim}
Expected : 1.000000e-008
Naive method : 9.094947e-009 (error=9.050530e-002)
Scilab method : 1.000000e-008 (error=1.654361e-016)
\end{verbatim}

The result is surprising, since the naive root has 
no correct digit and a relative error which is 14 orders 
of magnitude greater than the relative error of the Scilab root.

The explanation for such a behaviour is that the expression of the 
positive root is the following 

\begin{eqnarray}
x_+ &=& \frac{-1/\epsilon+ \sqrt{1/\epsilon^2+4\epsilon^2}}{2\epsilon}
\end{eqnarray}

and is numerically evalutated as 

\begin{verbatim}
\sqrt{1/\epsilon^2+4\epsilon^2} = 10000.000000000001818989
\end{verbatim}

As we see, the first digits are correct, but the last digits 
are polluted with rounding errors. When the expression $-1/\epsilon+ \sqrt{1/\epsilon^2+4\epsilon^2}$
is evaluated, the following computations are performed~:

\begin{verbatim}
-1/\epsilon+ \sqrt{1/\epsilon^2+4\epsilon^2} 
  = -10000.0 + 10000.000000000001818989 
  = 0.0000000000018189894035
\end{verbatim}

The user may think that the result is extreme, but it 
is not. Reducing furter the value of $\epsilon$ down to 
$\epsilon=10^{-11}$, we get the following output :

\begin{verbatim}
Expected : 1.000000e-022
Naive method : 0.000000e+000 (error=1.000000e+000)
Scilab method : 1.000000e-022 (error=1.175494e-016)
\end{verbatim}

The relative error is this time 16 orders of magnitude 
greater than the relative error of the Scilab root.
In fact, the naive implementation computes a false root $x_+$ even for 
a value of epsilon equal to $\epsilon=10^-3$, where the relative 
error is 7 times greater than the relative error produced by the 
\emph{roots} primitive.

\subsubsection{Overflow}

In this section, we analyse the overflow exception which is  
appearing when the discriminant of the quadratic equation 
is such that $b^2>> 4ac$.
We consider the following quadratic equation 
\begin{eqnarray}
x^2 + (1/\epsilon)x + 1 = 0
\end{eqnarray}
with $\epsilon\rightarrow 0$.

The roots of this equation are 
\begin{eqnarray}
x_- &\approx& -1/\epsilon \rightarrow -\infty, \qquad \epsilon \rightarrow 0\\
x_+ &\approx& -\epsilon \rightarrow 0^-, \qquad \epsilon \rightarrow 0
\end{eqnarray}
To create a difficult case, we search $\epsilon$ so that 
$1/\epsilon^2 = 10^{310}$, because we know that $10^{308}$
is the maximum value available with double precision floating 
point numbers. One possible solution is $\epsilon=10^{-155}$.

The following Scilab script shows an example of the computation
of the roots of such a polynomial with the \emph{roots}
primitive and with a naive implementation.

\lstset{language=Scilab}
\lstset{numbers=left}
\lstset{basicstyle=\footnotesize}
\lstset{keywordstyle=\bfseries}
\begin{lstlisting}
// Test #3 : overflow because of b
e=1.e-155
a = 1;
b = 1/e;
c = 1;
p=poly([c b a],"x","coeff");
expected = [-e;-1/e];
roots1 = myroots(p);
roots2 = roots(p);
error1 = abs(roots1-expected)/norm(expected);
error2 = abs(roots2-expected)/norm(expected);
printf("Expected : %e %e\n", expected(1),expected(2));
printf("Naive method : %e %e (error=%e)\n", roots1(1),roots1(2),error1);
printf("Scilab method : %e %e (error=%e)\n", roots2(1),roots2(2), error2);
\end{lstlisting}

The script then prints out :

\begin{verbatim}
Expected : -1.000000e-155 -1.000000e+155
Naive method : Inf Inf (error=Nan)
Scilab method : -1.000000e-155 -1.000000e+155 (error=0.000000e+000)
\end{verbatim}

As we see, the $b^2-4ac$ term has been evaluated as $1/\epsilon^2-4$,
which is approximately equal to $10^{310}$. This number cannot 
be represented in a floating point number. It therefore produces the 
IEEE overflow exception and set the result as \emph{Inf}.

\subsection{Explanations}

The following tricks are extracted from the 
\emph{quad} routine of the \emph{RPOLY} algorithm by
Jenkins \cite{Jenkins1975}. This algorithm is used by Scilab in the 
roots primitive, where a special case is handled when the 
degree of the equation is equal to 2, i.e. a quadratic equation.

\subsubsection{Properties of the roots}

One can easily show that the sum and the product of the roots
allow to recover the coefficients of the equation which was solve.
One can show that 
\begin{eqnarray}
x_- + x_+ &=&\frac{-b}{a}\\
x_- x_+ &=&\frac{c}{a}
\end{eqnarray}
Put in another form, one can state that the computed roots are 
solution of the normalized equation 
\begin{eqnarray}
x^2 - \left(\frac{x_- + x_+}{a}\right) x  + x_- x_+ &=&0
\end{eqnarray}

Other transformation leads to an alternative form for the roots. 
The original quadratic equation can be written as a quadratic 
equation on $1/x$
\begin{eqnarray}
c(1/x)^2 + b (1/x)  + a &=&0
\end{eqnarray}
Using the previous expressions for the solution of $ax^2+bx+c=0$ leads to the 
following expression of the roots of the quadratic equation when the 
discriminant is positive 
\begin{eqnarray}
x_- &=& \frac{2c}{-b+ \sqrt{b^2-4ac}}, \label{real:x-inverse}\\
x_+ &=& \frac{2c}{-b- \sqrt{b^2-4ac}} \label{real:x+inverse}
\end{eqnarray}
These roots can also be computed from \ref{real:x-}, with the 
multiplication by $-b+ \sqrt{b^2-4ac}$.

\subsubsection{Conditionning of the problem}

The conditionning of the problem may be evaluated with the 
computation of the partial derivatives of the roots of the 
equations with respect to the coefficients.
These partial derivatives measure the sensitivity of the 
roots of the equation with respect to small errors which might 
pollute the coefficients of the quadratic equations.

In the following, we note $x_-=\frac{-b- \sqrt{\Delta}}{2a}$ 
and $x_+=\frac{-b+ \sqrt{\Delta}}{2a}$ when $a\neq 0$.
If the discriminant is stricly positive and $a\neq 0$, i.e. if the roots 
of the quadratic are real, the partial derivatives of the 
roots are the following :
\begin{eqnarray}
\frac{\partial x_-}{\partial a} &=& \frac{c}{a\sqrt{\Delta}} + \frac{b+\sqrt{\Delta}}{2a^2}, \qquad a\neq 0, \qquad \Delta\neq 0\\
\frac{\partial x_+}{\partial a} &=& -\frac{c}{a\sqrt{\Delta}} + \frac{b-\sqrt{\Delta}}{2a^2}\\
\frac{\partial x_-}{\partial b} &=& \frac{-1-b/\sqrt{\Delta}}{2a}\\
\frac{\partial x_+}{\partial b} &=& \frac{-1+b/\sqrt{\Delta}}{2a}\\
\frac{\partial x_-}{\partial c} &=& \frac{1}{\sqrt{\Delta}}\\
\frac{\partial x_+}{\partial c} &=& -\frac{1}{\sqrt{\Delta}}
\end{eqnarray}

If the discriminant is zero, the partial derivatives of the 
double real root are the following :
\begin{eqnarray}
\frac{\partial x_\pm}{\partial a} &=& \frac{b}{2a^2}, \qquad a\neq 0\\
\frac{\partial x_\pm}{\partial b} &=& \frac{-1}{2a}\\
\frac{\partial x_\pm}{\partial c} &=& 0
\end{eqnarray}

The partial derivates indicate that if $a\approx 0$ or $\Delta\approx 0$,
the problem is ill-conditionned. 



\subsubsection{Floating-Point implementation : fixing rounding error}

In this section, we show how to compute the roots of a 
quadratic equation with protection against rounding 
errors, protection against overflow and a minimum 
amount of multiplications and divisions.

Few but important references deals with floating point
implementations of the roots of a quadratic polynomial.
These references include the important paper \cite{WhatEveryComputerScientist} by Golberg, 
the Numerical Recipes \cite{NumericalRecipes}, chapter 5, section 5.6
and \cite{FORSYTHE1991}, \cite{Nievergelt2003}, \cite{Kahan2004}.

The starting point is the mathematical solution of the quadratic equation, 
depending on the sign of the discriminant $\Delta=b^2 - 4ac$ :
\begin{itemize}
\item If $\Delta> 0$, there are two real roots, 
\begin{eqnarray}
x_\pm &=& \frac{-b\pm \sqrt{\Delta}}{2a}, \qquad a\neq 0
\end{eqnarray}
\item If $\Delta=0$, there are one double root,
\begin{eqnarray}
x_\pm &=& -\frac{b}{2a}, \qquad a\neq 0
\end{eqnarray}
\item If $\Delta< 0$, 
\begin{eqnarray}
x_\pm &=&\frac{-b}{2a} \pm i \frac{\sqrt{-\Delta}}{2a}, \qquad a\neq 0
\end{eqnarray}
\end{itemize}


In the following, we make the hypothesis that $a\neq 0$.

The previous experiments suggest that the floating point implementation
must deal with two different problems :
\begin{itemize}
\item rounding errors when $b^2\approx 4ac$ because of the cancelation of the 
terms which have opposite signs,
\item overflow in the computation of the discriminant $\Delta$ when $b$ is 
large in magnitude with respect to $a$ and $c$.
\end{itemize}

When $\Delta>0$, the rounding error problem can be splitted in two cases
\begin{itemize}
\item if $b<0$, then $-b+\sqrt{b^2-4ac}$ may suffer of rounding errors,
\item if $b>0$, then $-b-\sqrt{b^2-4ac}$ may suffer of rounding errors.
\end{itemize}
 
Obviously, the rounding problem will not appear when $\Delta<0$,
since the complex roots do not use the sum $-b+\sqrt{b^2-4ac}$.
When $\Delta=0$, the double root does not cause further trouble.
The rounding error problem must be solved only when $\Delta>0$ and the 
equation has two real roots.

A possible solution may found in combining the following expressions for the 
roots 
\begin{eqnarray}
x_- &=& \frac{-b- \sqrt{b^2-4ac}}{2a}, \label{real:x-2}\\
x_- &=& \frac{2c}{-b+ \sqrt{b^2-4ac}}, \label{real:x-inverse2}\\
x_+ &=& \frac{-b+ \sqrt{b^2-4ac}}{2a}, \label{real:x+2}\\
x_+ &=& \frac{2c}{-b- \sqrt{b^2-4ac}} \label{real:x+inverse2}
\end{eqnarray}

The trick is to pick the formula so that the sign of $b$ is the 
same as the sign of the square root.

The following choice allow to solve the rounding error problem 
\begin{itemize}
\item compute $x_-$ : if $b<0$, then compute $x_-$ from \ref{real:x-inverse2}, else 
(if $b>0$), compute $x_-$ from \ref{real:x-2},
\item compute $x_+$ : if $b<0$, then compute $x_+$ from \ref{real:x+2}, else 
(if $b>0$), compute $x_+$ from \ref{real:x+inverse2}.
\end{itemize}

The solution of the rounding error problem  can be adressed, by considering the 
modified Fagnano formulas
\begin{eqnarray}
x_1 &=& -\frac{2c}{b+sgn(b)\sqrt{b^2-4ac}}, \\
x_2 &=& -\frac{b+sgn(b)\sqrt{b^2-4ac}}{2a}, 
\end{eqnarray}
where 
\begin{eqnarray}
sgn(b)=\left\{\begin{array}{l}
1, \textrm{ if } b\geq 0,\\
-1, \textrm{ if } b< 0,
\end{array}\right.
\end{eqnarray}
The roots $x_{1,2}$ correspond to $x_{+,-}$ so that if $b<0$, $x_1=x_-$ and
if $b>0$, $x_1=x_+$. On the other hand, if $b<0$, $x_2=x_+$ and
if $b>0$, $x_2=x_-$.

An additionnal remark is that the division by two (and the multiplication
by 2) is exact with floating point numbers so these operations
cannot be a source of problem. But it is 
interesting to use $b/2$, which involves only one division, instead
of the three multiplications $2*c$, $2*a$ and $4*a*c$.
This leads to the following expressions of the real roots 
\begin{eqnarray}
x_- &=& -\frac{c}{(b/2)+sgn(b)\sqrt{(b/2)^2-ac}}, \\
x_+ &=& -\frac{(b/2)+sgn(b)\sqrt{(b/2)^2-ac}}{a}, 
\end{eqnarray}
which can be simplified into
\begin{eqnarray}
b'&=&b/2\\
h&=& -\left(b'+sgn(b)\sqrt{b'^2-ac}\right)\\
x_1 &=& \frac{c}{h}, \\
x_2 &=& \frac{h}{a}, 
\end{eqnarray}
where the discriminant is positive, i.e. $b'^2-ac>0$.

One can use the same value $b'=b/2$ with the complex roots in the 
case where the discriminant is negative, i.e. $b'^2-ac<0$ :
\begin{eqnarray}
x_1 &=& -\frac{b'}{a} - i \frac{\sqrt{ac-b'^2}}{a}, \\
x_2 &=& -\frac{b'}{a} + i \frac{\sqrt{ac-b'^2}}{a}, 
\end{eqnarray}

A more robust algorithm, based on the previous analysis is presented in figure \ref{robust-quadratic}.
By comparing \ref{naive-quadratic} and \ref{robust-quadratic}, we can see that 
the algorithms are different in many points.

\begin{figure}[htbp]
\begin{algorithmic}
\IF {$a=0$}
        \IF {$b=0$}
            \STATE $x_-\gets 0$
            \STATE $x_+\gets 0$
        \ELSE
            \STATE $x_-\gets -c/b$
            \STATE $x_+\gets 0$
        \ENDIF
\ELSIF {$c=0$}
        \STATE $x_-\gets -b/a$
        \STATE $x_+\gets 0$        
\ELSE
        \STATE $b'\gets b/2$
        \STATE $\Delta\gets b'^2 - ac$
        \IF {$\Delta<0$}
                \STATE $s\gets \sqrt{-\Delta}$
                \STATE $x_1^R\gets -b'/a$
                \STATE $x_1^I\gets -s/a$
                \STATE $x_2^R\gets x_-^R$
                \STATE $x_2^I\gets -x_1^I$
        \ELSIF {$\Delta=0$}
                \STATE $x_1\gets -b'/a$
                \STATE $x_2\gets x_2$
        \ELSE
                \STATE $s\gets \sqrt{\Delta}$
                \IF {$b>0$}
                    \STATE $g=1$
                \ELSE
                    \STATE $g=-1$
                \ENDIF
                \STATE $h=-(b'+g*s)$
                \STATE $x_1\gets c/h$
                \STATE $x_2\gets h/a$
        \ENDIF
\ENDIF 
\end{algorithmic}
\caption{A more robust algorithm to compute the roots of a quadratic equation}
\label{robust-quadratic}
\end{figure}

\subsubsection{Floating-Point implementation : fixing overflow problems}

The remaining problem is to compute $b'^2-ac$ without creating 
unnecessary overflows. 

Notice that a small improvment
has allread been done : if $|b|$ is close to the upper bound $10^{154}$, 
then $|b'|$ may be less difficult to process since $|b'|=|b|/2 < |b|$.
One can then compute the square root by using normalization methods, 
so that the overflow problem can be drastically reduced.
The method is based on the fact that the term $b'^2-ac$ can be 
evaluted with two equivalent formulas
\begin{eqnarray}
b'^2-ac &=& b'^2\left[1-(a/b')(c/b')\right] \\
b'^2-ac &=& c\left[b'(b'/c) - a\right]
\end{eqnarray}

\begin{itemize}
\item If $|b'|>|c|>0$, then the expression involving $\left(1-(a/b')(c/b')\right)$
is so that no overflow is possible since $|c/b'| < 1$ and the problem occurs
only when $b$ is large in magnitude with respect to $a$ and $c$.
\item If $|c|>|b'|>0$, then the expression involving $\left(b'(b'/c) - a\right)$
should limit the possible overflows since $|b'/c| < 1$.
\end{itemize}
These normalization tricks are similar to the one used by Smith in the 
algorithm for the division of complex numbers \cite{Smith1962}.

\subsection{References}

The 1966 technical report by G. Forsythe \cite{Forsythe1966} 
presents the floating point system and the possible large error 
in using mathematical algorithms blindly. An accurate way of solving 
a quadratic is outlined. A few general remarks are made about 
computational mathematics. The 1991 paper by Goldberg 
\cite{WhatEveryComputerScientist} is a general presentation of the floating
point system and its consequences. It begins with background on floating point 
representation and rounding errors, continues with a discussion
of the IEEE floating point standard and concludes with examples of how
computer system builders can better support floating point. The section
1.4, "Cancellation" specificaly consider the computation of the roots
of a quadratic equation.
One can also consult the experiments performed by Nievergelt in \cite{Nievergelt2003}.


