% Copyright (C) 2008-2010 - Consortium Scilab - Digiteo - Michael Baudin
%
% This file must be used under the terms of the 
% Creative Commons Attribution-ShareAlike 3.0 Unported License :
% http://creativecommons.org/licenses/by-sa/3.0/

%
% scilabisnotnaive.tex --
%   Some notes about floating point issues in Scilab.
%
\documentclass[12pt]{article}

\include{macros}

\begin{document}
\author{Michael Baudin}
\date{January 2010}
\title{Scilab is not naive}
\maketitle
\begin{abstract}
Most of the time, the mathematical formula is directly used in the 
Scilab source code. But, in many algorithms, some additional work is 
performed, which takes into account the fact that the computer does not 
process mathematical real values, but performs computations with their 
floating point representation. The goal of this article is to show that, 
in many situations, Scilab is not naive and use algorithms which have been 
specifically tailored for floating point computers. We analyze in this 
article the particular case of the quadratic equation, the complex 
division and the numerical derivatives. In each example, we show that 
the naive algorithm is not sufficiently accurate, while Scilab 
implementation is much more robust. 
\end{abstract}

\tableofcontents

% Copyright (C) 2008-2010 - Consortium Scilab - Digiteo - Michael Baudin
%
% This file must be used under the terms of the 
% Creative Commons Attribution-ShareAlike 3.0 Unported License :
% http://creativecommons.org/licenses/by-sa/3.0/

\section{Introduction}

As a practical example of the problem considered in this
document, consider the following numerical experiments. 
The following session is an example of 
a Scilab session, where we compute the real number 0.1 by two different, 
but mathematically equivalent ways.
\lstset{language=scilabscript}
\begin{lstlisting}
-->format(25)
-->0.1
 ans  =
    0.1000000000000000055511  
-->1.0-0.9
 ans  =
    0.0999999999999999777955  
-->0.1 == 1.0 - 0.9
 ans  =
  F  
\end{lstlisting}

I guess that for a person who has never heard of these problems,
this experiment may be a shock. To get things clearer, let's 
check that the sinus function is also approximated in the sense that 
the value of $sin(\pi)$ is \emph{not exactly} zero.
\lstset{language=scilabscript}
\begin{lstlisting}
-->format(25)
-->sin(0.0)
 ans  =
    0.  
-->sin(%pi)
 ans  =
    0.0000000000000001224647  
\end{lstlisting}

\index{floating point numbers}
With symbolic computation systems, such as Maple\cite{WWWMapleSoft}, 
Mathematica\cite{WWWMathematica} or Maxima\cite{WWWMaxima} for example, 
the calculations are performed with abstract mathematical 
symbols. Therefore, there is no loss of accuracy, as long as 
no numerical evaluation is performed. If a numerical solution is 
required as a rational number of the form $p/q$ where $p$ and $q$ are 
integers and $q\neq 0$, there is still no loss of accuracy.
On the other hand, in numerical computing systems, such as 
Scilab\cite{WWWScilab}, Matlab\cite{WWWMatlab} or Octave\cite{WWWOctave} for example, 
the computations are performed with floating point numbers.
When a numerical value is stored, it is generally associated with a 
rounding error. 

The difficulty of numerical computations is generated by the fact that, while 
the mathematics treat with \emph{real} numbers, the 
computer deals with their \emph{floating point representations}.
This is the difference between the 
\emph{naive}, mathematical, approach, and the \emph{numerical},
floating-point, implementation.

In this article, we will not present the floating point arithmetic in 
detail. Instead, we will show examples of floating point issues by 
using the following algebraic and experimental approach.
\begin{enumerate}
\item First, we will derive the basic theory of a mathematical formula. 
\item Then, we will implement it in Scilab and compare with the 
result given by the equivalent function provided by Scilab.
As we will see, some particular cases do not work well
with our formula, while the Scilab function computes a correct
result.
\item Finally, we will analyze the \emph{reasons} of the differences.
\end{enumerate}
Our numerical experiments will be based on Scilab.

\index{relative error}
\index{absolute error}
In order to measure the accuracy of the results, we will use 
two different measures of error: the relative error and the 
absolute error\cite{Higham:2002:ASN}. 
Assume that $x_c\in\RR$ is a computed value and 
$x_e\in\RR$ is the expected (exact) value. We are looking for 
a measure of the \emph{difference} between these two real numbers. 
Most of the time, we use the relative error 
\begin{eqnarray}
e_r=\frac{|x_c-x_e|}{|x_e|},
\end{eqnarray}
where we assume that $x_e\neq 0$. The relative error $e_r$ is linked with the number of significant 
digits in the computed value $x_c$. For example, if the relative 
error $e_r=10^{-6}$, then the number of significant digits is 6.

When the expected value is zero, the relative error cannot 
be computed, and we then use instead the absolute error  
\begin{eqnarray}
e_a=|x_c-x_e|.
\end{eqnarray}

A practical way of checking the expected result of a computation
is to compare the formula computed "by hand" with the result 
produced by a symbolic tool. Recently, Wolfram has launched the 
\url{http://www.wolframalpha.com}
website which let us access to Mathematica with a classical web browser.
Many examples in this document have been validated with this tool.

\index{IEEE 754}
In the following, we make a brief overview of floating point numbers used in Scilab.
Real variables in Scilab are stored in 
\emph{double precision} floating point variables. Indeed, Scilab uses the IEEE 754 standard so that real 
variables are stored with 64 bits floating point numbers, called \emph{doubles}.
The floating point number associated with a given $x\in\RR$ will be 
denoted by $fl(x)$.

While the real numbers form a continuum, floating point numbers 
are both finite and bounded. Not all real numbers can be represented by a floating point number.
Indeed, there is a infinite number of reals, while there is a finite 
number of floating point numbers. 
In fact, there are, at most, $2^{64}$ different 64 bits floating point numbers. 
This leads to \emph{roundoff}, \emph{underflow} and \emph{overflow}.

The double floating point numbers are associated with a machine epsilon equal to $2^{-52}$,
which is approximately equal to $10^{-16}$. This parameter is 
stored in the \scivar{\%eps} Scilab variable. Therefore, we can expect, at best, 
approximately 16 significant decimal digits. 
This parameter does not depend on the machine we use. Indeed, be it a Linux or a Windows system,
Scilab uses IEEE doubles. Therefore, the value of the \scivar{\%eps} variable is always the same
in Scilab.

Negative normalized floating point numbers are in the 
range $[-10^{308},-10^{-307}]$ and positive normalized floating point numbers are in the 
range $[10^{-307},10^{308}]$. The limits given in the previous intervals 
are only decimal approximations. Any real number greater than $10^{309}$
or smaller than $-10^{309}$ is not representable as a double and is stored with the 
"infinite" value: in this case, we say that an overflow occurred. 
A real which magnitude is smaller than $10^{-324}$ is not representable as a 
double and is stored as a zero: in this case, we say that an underflow occurred.

The outline of this paper is the following. In the first section,
we compute the roots of a quadratic equation. In the second section,
we compute the numerical derivatives of a function. In the final section,
we perform a numerically difficult division, with complex numbers.
The examples presented in this introduction are presented in the 
appendix of this document.



% Copyright (C) 2008-2010 - Consortium Scilab - Digiteo - Michael Baudin
%
% This file must be used under the terms of the 
% Creative Commons Attribution-ShareAlike 3.0 Unported License :
% http://creativecommons.org/licenses/by-sa/3.0/

\section{Quadratic equation}

In this section, we analyze the computation of the roots of a quadratic polynomial.
As we shall see, there is a whole \emph{world} from the mathematical formulas to the 
implementation of such computations. In the first part, we briefly report the formulas which allow to 
compute the real roots of a quadratic equation with real coefficients.
We then present the naive algorithm based on these mathematical formulas. 
In the second part, we make some experiments in Scilab and compare our
naive algorithm with the \scifun{roots} Scilab function.
In the third part, we analyze why and how floating point numbers must be 
taken into account when the roots of a quadratic are required.

\subsection{Theory}

In this section, we present the mathematical formulas which allow to compute the 
real roots of a quadratic polynomial with real coefficients.
We chose to begin by the example of a quadratic equation, because most of 
us exactly know (or \emph{think} we know) how to solve such an equation with a computer.

Assume that $a, b, c \in \RR$ are given coefficients and $a\neq 0$. 
Consider the following quadratic \cite{wikipediaquadratic,wikipedialossofsign,mathworldquadratic} equation:
\begin{eqnarray}
\label{eq-quadratic}
a x^2 + b x + c = 0,
\end{eqnarray}
where $x\in\RR$ is the unknown.

Let us define by $\Delta=b^2-4ac$ the discriminant of the quadratic equation.
We consider the mathematical solution of the quadratic equation, 
depending on the sign of the discriminant $\Delta=b^2 - 4ac$.
\begin{itemize}
\item If $\Delta> 0$, there are two real roots: 
% Keep two equations for the root selection explanation
\begin{eqnarray}
x_- &=& \frac{-b- \sqrt{\Delta}}{2a}, \label{real:x-} \\
x_+ &=& \frac{-b+ \sqrt{\Delta}}{2a}. \label{real:x+}
\end{eqnarray}
\item If $\Delta=0$, there is one double root:
\begin{eqnarray}
\label{realdouble:x-+}
x_\pm &=& -\frac{b}{2a}.
\end{eqnarray}
\item If $\Delta< 0$, there are two complex roots:
\begin{eqnarray}
\label{complex:x-+}
x_\pm &=&\frac{-b}{2a} \pm i \frac{\sqrt{-\Delta}}{2a}.
\end{eqnarray}
\end{itemize}

We now consider a simplified algorithm where we only compute the real roots of the 
quadratic, assuming that $\Delta>0$.
This naive algorithm is presented in figure \ref{naive-quadratic}.

\begin{algorithm}[htbp]
\SetKwInOut{Input}{input}\SetKwInOut{Output}{output}
\Input{$a,b,c$}
\Output{$x_-$, $x_+$}
$\Delta:= b^2-4ac$\;
$s:= \sqrt{\Delta}$\;
$x_-:= (-b-s)/(2a)$\;
$x_+:= (-b+s)/(2a)$\;
\caption{Naive algorithm to compute the real roots of a quadratic equation. - We assume that $\Delta> 0$.}
\label{naive-quadratic}
\end{algorithm}

%%%%%%%%%%%%%%%%%%%%%%%%%%%%%%%%%%%%%%%%%%%%%%%%%%%%%%%%%%%%%%%%%
\subsection{Experiments}

In this section, we compare our naive algorithm with the \scifun{roots} function.
We begin by defining a function which naively implements the mathematical formulas.
Then we use our naive function on two particular examples. 
In the first example, we focus on massive cancellation and in the second example, we 
focus on overflow problems.

The following Scilab function \scifun{myroots} is a straightforward implementation
of the previous formulas. It takes as input the coefficients of the quadratic, stored in the 
vector variable \scivar{p}, and returns the two roots in the vector \scivar{r}.
\lstset{language=scilabscript}
\begin{lstlisting}
function r=myroots(p)
  c=coeff(p,0);
  b=coeff(p,1);
  a=coeff(p,2);
  r(1)=(-b+sqrt(b^2-4*a*c))/(2*a);
  r(2)=(-b-sqrt(b^2-4*a*c))/(2*a);
endfunction
\end{lstlisting}

%%%%%%%%%%%%%%%%%%%%%%%%%%%%%%%%%%%%%%%%%%%%%%%%%%%%%%%%%%%%%%%%%
\subsubsection{Massive cancellation}
\label{section-exp-quadraticrounding}

\index{cancellation}
\index{massive cancellation}
We analyze the rounding errors which are 
appearing when the discriminant of the quadratic equation 
is such that $b^2\gg 4ac$.
We consider the following quadratic equation 
\begin{eqnarray}
\label{sinn-eq-roundingerror}
\epsilon x^2 + (1/\epsilon)x - \epsilon = 0
\end{eqnarray}
with $\epsilon>0$. For example, consider the special case $\epsilon=0.0001=10^{-4}$. 
The discriminant of this equation is $\Delta = 1/\epsilon^2+4\epsilon^2$.

The two real solutions of the quadratic equation are
\begin{eqnarray}
\label{sinn-eq-roundingerror-roots}
x_- = \frac{-1/\epsilon- \sqrt{1/\epsilon^2+4\epsilon^2}}{2\epsilon}, \qquad
x_+ = \frac{-1/\epsilon+ \sqrt{1/\epsilon^2+4\epsilon^2}}{2\epsilon}.
\end{eqnarray}
These roots are approximated by 
\begin{eqnarray}
\label{sinn-eq-roundingerror-roots-approx}
x_- \approx  -1/\epsilon^2, \qquad
x_+ \approx  \epsilon^2,
\end{eqnarray}
when $\epsilon$ is close to zero.
We now consider the limit of the two roots when $\epsilon \rightarrow 0$. We have 
\begin{eqnarray}
\lim_{\epsilon\rightarrow 0} x_- = -\infty, \qquad
\lim_{\epsilon\rightarrow 0} x_+ = 0.
\end{eqnarray}

\index{\scifun{roots}}
In the following Scilab script, we compare the roots computed by the \scifun{roots}
function and the roots computed by our naive function. We begin by creating 
a polynomial with the \scifun{poly} function, which is given the coefficients of 
the polynomial. Only the positive root $x_+ \approx \epsilon^2$ is considered in this 
test. Indeed, the $x_-$ root is so that $x_- \rightarrow -\infty$ in both 
implementations.
\lstset{language=scilabscript}
\begin{lstlisting}
p=poly([-0.0001 10000.0 0.0001],"x","coeff");
e1 = 1e-8;
roots1 = myroots(p);
r1 = roots1(1);
roots2 = roots(p);
r2 = roots2(1);
error1 = abs(r1-e1)/e1;
error2 = abs(r2-e1)/e1;
printf("Expected : %e\n", e1);
printf("Naive method : %e (error=%e)\n", r1,error1);
printf("Scilab method : %e (error=%e)\n", r2, error2);
\end{lstlisting}

The previous script produces the following output.
\lstset{language=scilabscript}
\begin{lstlisting}
Expected : 1.000000e-008
Naive method : 9.094947e-009 (error=9.050530e-002)
Scilab method : 1.000000e-008 (error=1.654361e-016)
\end{lstlisting}

We see that the naive method produces a root which has no significant digit 
and a relative error which is 14 orders of magnitude greater than the relative 
error of the Scilab root.

This behavior is explained by the fact that the expression for the 
positive root $x_+$ given by the equality \ref{real:x+} is numerically evaluated 
as following. We first consider how the discriminant $\Delta = 1/\epsilon^2+4\epsilon^2$ 
is computed. The term $1/\epsilon^2$ is equal to 100000000 and the term $4\epsilon^2$ 
is equal to 0.00000004. Therefore, the sum of these two terms is equal to 100000000.000000045.
Hence, the square root of the discriminant is 
\begin{eqnarray}
\sqrt{1/\epsilon^2+4\epsilon^2} = 10000.000000000001818989.
\end{eqnarray}
As we see, the first digits are correct, but the last digits 
are subject to rounding errors. When the expression $-1/\epsilon+ \sqrt{1/\epsilon^2+4\epsilon^2}$
is evaluated, the following computations are performed~:
\begin{eqnarray}
-1/\epsilon+ \sqrt{1/\epsilon^2+4\epsilon^2} &=& -10000.0 + 10000.000000000001818989 \\
  &=& 0.0000000000018189894035
\end{eqnarray}
We see that the result is mainly driven by the cancellation of significant digits.

We may think that the result is extreme, but it 
is not. For example, consider the case where we reduce further the value of $\epsilon$ down to 
$\epsilon=10^{-11}$, we get the following output :
\begin{lstlisting}
Expected : 1.000000e-022
Naive method : 0.000000e+000 (error=1.000000e+000)
Scilab method : 1.000000e-022 (error=1.175494e-016)
\end{lstlisting}

The relative error is this time 16 orders of magnitude 
greater than the relative error of the Scilab root.
There is no significant decimal digit in the result. 
In fact, the naive implementation computes a false root $x_+$ even for 
a value of epsilon equal to $\epsilon=10^{-3}$, where the relative 
error is 7 orders of magnitude greater than the relative error produced by the 
\scifun{roots} function.

%%%%%%%%%%%%%%%%%%%%%%%%%%%%%%%%%%%%%%%%%%%%%%%%%%%%%%%%%%%%%%%%%
\subsubsection{Overflow}
\label{section-exp-quadraticoverflow}

\index{overflow}
In this section, we analyse the overflow which appears  
when the discriminant of the quadratic equation 
is such that $b^2- 4ac$ is not representable as a double.
We consider the following quadratic equation 
\begin{eqnarray}
\label{sinn-eq-overflowerror}
x^2 + (1/\epsilon)x + 1 = 0
\end{eqnarray}
with $\epsilon>0$. We especially consider the case $\epsilon\rightarrow 0$. 
The discriminant of this equation is $\Delta=1/\epsilon^2 -4$. Assume that the discriminant is positive.
Therefore, the roots of the quadratic equation are 
\begin{eqnarray}
x_- = \frac{-1/\epsilon- \sqrt{1/\epsilon^2-4}}{2}, \qquad
x_+ = \frac{-1/\epsilon+ \sqrt{1/\epsilon^2-4}}{2}.
\end{eqnarray}
These roots are approximated by 
\begin{eqnarray}
\label{sinn-eq-rootsoverflowapprox}
x_- \approx  -1/\epsilon, \qquad
x_+ \approx  -\epsilon,
\end{eqnarray}
when $\epsilon$ is close to zero.
We now consider the limit of the two roots when $\epsilon \rightarrow 0$. We have 
\begin{eqnarray}
\lim_{\epsilon\rightarrow 0} x_- = -\infty, \qquad
\lim_{\epsilon\rightarrow 0} x_+ = 0^-.
\end{eqnarray}
To create a difficult case, we search $\epsilon$ so that 
$1/\epsilon^2 > 10^{308}$, because we know that $10^{308}$
is the maximum representable double precision floating 
point number. Therefore, we expect that something should go wrong 
in the computation of the expression $\sqrt{1/\epsilon^2-4}$. 
We choose $\epsilon=10^{-155}$.

In the following script, we compare the roots computed by the \scifun{roots}
function and our naive implementation.
\lstset{language=scilabscript}
\begin{lstlisting}
e=1.e-155
a = 1;
b = 1/e;
c = 1;
p=poly([c b a],"x","coeff");
expected = [-e;-1/e];
roots1 = myroots(p);
roots2 = roots(p);
error1 = abs(roots1-expected)/norm(expected);
error2 = abs(roots2-expected)/norm(expected);
printf("Expected : %e %e\n", expected(1),expected(2));
printf("Naive method : %e %e (error=%e %e)\n", ...
  roots1(1),roots1(2), error1(1),error1(2));
printf("Scilab method : %e %e (error=%e %e)\n", ...
  roots2(1),roots2(2), error2(1),error2(2));
\end{lstlisting}

The previous script produces the following output.
\begin{lstlisting}
Expected : -1.000000e-155 -1.000000e+155
Naive method : Inf Inf (error=Nan Nan)
Scilab method : -1.000000e-155 -1.000000e+155 
            (error=0.000000e+000 0.000000e+000)
\end{lstlisting}

In this case, the discriminant $\Delta = b^2-4ac$ has been evaluated as $1/\epsilon^2-4$,
which is approximately equal to $10^{310}$. This number cannot 
be represented in a double precision floating point number. It therefore produces the 
IEEE Infinite number, which is displayed by Scilab as \scivar{Inf}.
The Infinite number is associated with an algebra and functions can perfectly 
take this number as input. Therefore, when the square root function 
must compute $\sqrt{\Delta}$, it produces again \scivar{Inf}. This number is 
then propagated into the final roots.

%%%%%%%%%%%%%%%%%%%%%%%%%%%%%%%%%%%%%%%%%%%%%%%%%%%%%%%%%%%%%%%%%
\subsection{Explanations}

In this section, we suggest robust methods to compute the roots of a 
quadratic equation. 

\index{Jenkins, M. A.}
\index{Traub, J. F.}
The methods presented in this section are extracted from the 
\emph{quad} routine of the \emph{RPOLY} algorithm by
Jenkins and Traub \cite{JenkinsTraub1970,Jenkins1975}. This algorithm is used by Scilab in the 
\scifun{roots} function, where a special case is used when the 
degree of the equation is equal to 2, i.e. a quadratic equation.

%%%%%%%%%%%%%%%%%%%%%%%%%%%%%%%%%%%%%%%%%%%%%%%%%%%%%%%%%%%%%%%%%
\subsubsection{Properties of the roots}

In this section, we present elementary results, which will be 
used in the derivation of robust floating point formulas of the roots 
of the quadratic equation. 

Let us assume that the quadratic equation \ref{eq-quadratic}, with 
real coefficients $a,b,c\in\RR$ and $a>0$ has a positive 
discriminant $\Delta=b^2-4ac$. Therefore, the two real roots of 
the quadratic equation are given by the equations \ref{real:x-} and \ref{real:x+}.
We can prove that the sum and the product of the roots satisfy the equations 
\begin{eqnarray}
\label{eq-rootsprops}
x_- + x_+ =\frac{-b}{a},\qquad
x_- x_+ =\frac{c}{a}.
\end{eqnarray}
Therefore, the roots are the solution of the normalized 
quadratic equation 
\begin{eqnarray}
\label{eq-rootsprops2}
x^2 - (x_- + x_+) x  + x_- x_+ &=&0.
\end{eqnarray}

Another transformation leads to an alternative form of the roots. 
Indeed, the original quadratic equation can be written as a quadratic 
equation of the unknown $1/x$. Consider the quadratic equation \ref{eq-quadratic}
and divide it by $1/x^2$, assuming that $x\neq 0$. This leads to the equation 
\begin{eqnarray}
\label{eq-quadraticinverse}
c(1/x)^2 + b (1/x)  + a &=&0,
\end{eqnarray}
where we assume that $x\neq 0$.
The two real roots of the quadratic equation \ref{eq-quadraticinverse} are 
\begin{eqnarray}
% Keep two equations for the root selection explanation
x_- &=& \frac{2c}{-b+ \sqrt{b^2-4ac}}, \label{real:x-inverse}\\
x_+ &=& \frac{2c}{-b- \sqrt{b^2-4ac}} \label{real:x+inverse}.
\end{eqnarray}
The expressions \ref{real:x-inverse} and \ref{real:x+inverse} can 
also be derived directly from the equations \ref{real:x-} and \ref{real:x+}.
For that purpose, it suffices to multiply their numerator and denominator 
by $-b+ \sqrt{b^2-4ac}$.

%%%%%%%%%%%%%%%%%%%%%%%%%%%%%%%%%%%%%%%%%%%%%%%%%%%%%%%%%%%%%%%%%
\subsubsection{Floating-Point implementation : overview}

The numerical experiments presented in sections \ref{section-exp-quadraticrounding} and
\ref{section-exp-quadraticoverflow} suggest that the floating point implementation
must deal with two different problems:
\begin{itemize}
\item massive cancellation when $b^2\gg 4ac$ because of the cancellation of the 
terms $-b$ and $\pm\sqrt{b^2-4ac}$ which may have opposite signs,
\item overflow in the computation of the square root of the 
discriminant $\sqrt{\pm(b^2-4ac)}$ when $b^2-4ac$ is not representable as 
a floating point number.
\end{itemize}

The cancellation problem occurs only when the discriminant is positive, i.e. only
when there are two real roots. Indeed, the cancellation will not appear when $\Delta<0$,
since the complex roots do not use the sum $-b\pm\sqrt{b^2-4ac}$.
When $\Delta=0$, the double real root does not cause any trouble.
Therefore, we must take into account for the cancellation problem only in the 
equations \ref{real:x-} and \ref{real:x+}.

On the other hand, the overflow problem occurs whatever the sign of the discriminant
but does not occur when $\Delta=0$. Therefore, we must take into account 
for this problem in the equations \ref{real:x-}, \ref{real:x+} and \ref{complex:x-+}.
In section \ref{section-quadratic-fixcancellation}, we focus on the cancellation error while 
the overflow problem is addressed in section \ref{section-quadratic-fixoverflow}.

%%%%%%%%%%%%%%%%%%%%%%%%%%%%%%%%%%%%%%%%%%%%%%%%%%%%%%%%%%%%%%%%%
\subsubsection{Floating-Point implementation : fixing massive cancellation}
\label{section-quadratic-fixcancellation}
In this section, we present the computation of the roots of a 
quadratic equation with protection against massive cancellation. 

When the discriminant $\Delta$ is positive, the massive cancellation problem can be split in two cases:
\begin{itemize}
\item if $b<0$, then $-b-\sqrt{b^2-4ac}$ may suffer of massive cancellation
because $-b$ is positive and $-\sqrt{b^2-4ac}$ is negative,
\item if $b>0$, then $-b+\sqrt{b^2-4ac}$ may suffer of massive cancellation because 
$-b$ is negative and $\sqrt{b^2-4ac}$ is positive.
\end{itemize}
Therefore, 
\begin{itemize}
\item if $b>0$, we should use the expression $-b-\sqrt{b^2-4ac}$,
\item if $b<0$, we should use the expression $-b+\sqrt{b^2-4ac}$.
\end{itemize}
The solution consists in a combination of the following expressions of the 
roots given by, on one hand the equations \ref{real:x-} and \ref{real:x+},
and, on the other hand the equations \ref{real:x-inverse} and \ref{real:x+inverse}.
We pick the formula so that the sign of $b$ is the 
same as the sign of the square root.
The following choice allow to solve the massive cancellation problem:
\begin{itemize}
\item if $b<0$, then compute $x_-$ from \ref{real:x-inverse}, else 
(if $b>0$), compute $x_-$ from \ref{real:x-},
\item if $b<0$, then compute $x_+$ from \ref{real:x+}, else 
(if $b>0$), compute $x_+$ from \ref{real:x+inverse}.
\end{itemize}

We can also consider the modified Fagnano formulas
\begin{eqnarray}
x_1 &=& -\frac{2c}{b+sgn(b)\sqrt{b^2-4ac}}, \\
x_2 &=& -\frac{b+sgn(b)\sqrt{b^2-4ac}}{2a}, 
\end{eqnarray}
where the sign function is defined by 
\begin{eqnarray}
sgn(b)=\left\{\begin{array}{l}
1, \textrm{ if } b\geq 0,\\
-1, \textrm{ if } b< 0.
\end{array}\right.
\end{eqnarray}
The roots $x_{1,2}$ correspond to the roots $x_{+,-}$. Indeed, on one hand, 
if $b<0$, $x_1=x_-$ and if $b>0$, $x_1=x_+$. On the other hand, if $b<0$, $x_2=x_+$ and
if $b>0$, $x_2=x_-$.

Moreover, we notice that the division by two (and the multiplication
by 2) is exact with floating point numbers so these operations
cannot be a source of problem. But it is 
interesting to use $b/2$, which involves only one division, instead
of the three multiplications $2*c$, $2*a$ and $4*a*c$.
This leads to the following expressions of the real roots 
\begin{eqnarray}
x_1 &=& -\frac{c}{(b/2)+sgn(b)\sqrt{(b/2)^2-ac}}, \\
x_2 &=& -\frac{(b/2)+sgn(b)\sqrt{(b/2)^2-ac}}{a}.
\end{eqnarray}
Therefore, the two real roots can be computed by the following sequence of 
computations:
\begin{eqnarray}
b'&:=&b/2, \qquad \Delta' := b'^2-ac,\\
h&:=& -\left(b'+sgn(b)\sqrt{\Delta'}\right)\\
x_1 &:=& \frac{c}{h}, \qquad x_2 := \frac{h}{a}. 
\end{eqnarray}

In the case where the discriminant $\Delta' := b'^2-ac$ is negative, the 
two complex roots are
\begin{eqnarray}
x_1 &=& -\frac{b'}{a} - i \frac{\sqrt{ac-b'^2}}{a}, \qquad 
x_2 = -\frac{b'}{a} + i \frac{\sqrt{ac-b'^2}}{a}. 
\end{eqnarray}

A more robust algorithm, based on the previous analysis is presented in figure \ref{robust-quadratic}.
By comparing \ref{naive-quadratic} and \ref{robust-quadratic}, we can see that 
the algorithms are different in many points.

\begin{algorithm}[htbp]
\SetKwInOut{Input}{input}\SetKwInOut{Output}{output}
\Input{$a,b,c$}
\Output{$x_-^R$, $x_-^I$, $x_+^R$, $x_+^I$}
\uIf {$a=0$} {
        \eIf {$b=0$} {
             $x_-^R:= 0$ , $x_-^I:= 0$ \;
             $x_+^R:= 0$ , $x_+^I:= 0$ \;
        } {
             $x_-^R:= -c/b$ , $x_-^I:= 0$ \;
             $x_+^R:= 0$ , $x_+^I:= 0$ \;
        }
}
\Else {
         $b':= b/2$ \;
         $\Delta:= b'^2 - ac$ \;
        \uIf {$\Delta<0$} {
                 $s:= \sqrt{-\Delta}$ \;
                 $x_-^R:= -b'/a$ , $x_-^I:= -s/a$ \;
                 $x_+^R:= x_-^R$ , $x_+^I:= -x_1^I$ \;
        }
        \uElseIf {$\Delta=0$} {
                 $x_-:= -b'/a$ , $x_-^I:= 0$ \;
                 $x_+:= x_2$ , $x_+^I:= 0$ \;
        }
        \Else {
                 $s:= \sqrt{\Delta}$ \;
                \uIf {$b>0$}{
                     $g:=1$ \;
                }
                \Else {
                     $g:=-1$ \;
                }
                 $h:=-(b'+g*s)$ \;
                 $x_-^R:= c/h$ , $x_-^I:= 0$ \;
                 $x_+^R:= h/a$ , $x_+^I:= 0$ \;
        }
}
\caption{A more robust algorithm to compute the roots of a quadratic equation. This algorithm 
takes as input arguments the real coefficients $a,b,c$ and returns the real and imaginary parts 
of the two roots, i.e. returns $x_-^R$, $x_-^I$, $x_+^R$, $x_+^I$.}
\label{robust-quadratic}
\end{algorithm}


%%%%%%%%%%%%%%%%%%%%%%%%%%%%%%%%%%%%%%%%%%%%%%%%%%%%%%%%%%%%%%%%%
\subsubsection{Floating-Point implementation : fixing overflow problems}
\label{section-quadratic-fixoverflow}

The remaining problem is to compute the square root of the 
discriminant $\sqrt{\pm(b'^2-ac)}$ without creating 
unnecessary overflows. In order to simplify the discussion, we 
focus on the computation of $\sqrt{b'^2-ac}$.

Obviously, the problem occur for large values of $b'$. 
Notice that a (very) small improvement has already been done. Indeed, we have the 
inequality $|b'|=|b|/2<|b|$
so that overflows are twice less likely to occur. The current upper bound 
for $|b'|$ is $10^{154}$, which is associated with 
$b^{'2}\leq 10^{308}$, the maximum double value before overflow.
The goal is therefore to increase the possible range of values of $b'$ without 
generating unnecessary overflows.

Consider the situation when $b'$ is large in magnitude with respect to $a$ and $c$. 
In that case, notice that we first square $b'$ to get $b^{'2}$
and then compute the square root $\sqrt{b^{'2} - ac}$.
Hence, we can factor the expression by $b^{'2}$ and move this 
term outside the square root, which makes the term $|b'|$ appear.
This method allows to compute the expression $\sqrt{b^{'2} - ac}$,
without squaring $b'$ when it is not necessary.

In the general case, we use the fact that the term $b'^2-ac$ can be 
evaluated with the two following equivalent formulas:
\begin{eqnarray}
b'^2-ac &=& b'^2\left[1-(a/b')(c/b')\right], \label{eq-discroverflow1}\\
b'^2-ac &=& c\left[b'(b'/c) - a\right].\label{eq-discroverflow2}
\end{eqnarray}
The goal is then to compute the square root $s = \sqrt{b'^2-ac}$.
\begin{itemize}
\item If $|b'|>|c|>0$, then the equation \ref{eq-discroverflow1} involves the expression $1-(a/b')(c/b')$.
The term $1-(a/b')(c/b')$ is so that no overflow is possible since $|c/b'| < 1$ (the overflow problem occurs
only when $b$ is large in magnitude with respect to both $a$ and $c$).
In this case, we use the expression
\begin{eqnarray}
e &=& 1-(a/b')(c/b'),
\end{eqnarray}
and compute
\begin{eqnarray}
\label{quadratic-overflow-trick1}
s &=& \pm|b'|\sqrt{|e|}.
\end{eqnarray}
In the previous equation, we use the sign + when $e$ is positive
and the sign - when $e$ is negative.

\item If $|c|>|b'|>0$, then the equation \ref{eq-discroverflow2} involves the expression $b'(b'/c) - a$.
The term $b'(b'/c) - a$ should limit the possible overflows since $|b'/c| < 1$. This implies that 
$\left|b'(b'/c)\right|<|b'|$. (There is 
still a possibility of overflow, for example in the case where $b'(b'/c)$
is near, but under, the overflow limit and $a$ is large.)
Therefore, we use the expression
\begin{eqnarray}
e &=& b'(b'/c) - a,
\end{eqnarray}
and compute 
\begin{eqnarray}
\label{quadratic-overflow-trick2}
s &=& \pm\sqrt{|c|} \sqrt{|e|}.
\end{eqnarray}
In the previous equation, we use the sign + when $e$ is positive
and the sign - when $e$ is negative.
\end{itemize}
In both equations \ref{quadratic-overflow-trick1} and \ref{quadratic-overflow-trick2}, 
the parenthesis must be strictly used. This property is ensured by the IEEE standard 
and by the Scilab language. 
This normalization method are similar to the one used by Smith in the 
algorithm for the division of complex numbers \cite{Smith1962} and 
which will be reviewed in the next section.



%%%%%%%%%%%%%%%%%%%%%%%%%%%%%%%%%%%%%%%%%%%%%%%%%%%%%%%%%%%%%%%%%
\subsection{References}

\index{Forsythe, George}
The 1966 technical report by G. Forsythe \cite{Forsythe1966} 
presents the floating point system and the possible large error 
in using mathematical algorithms blindly. An accurate way of solving 
a quadratic is outlined. A few general remarks are made about 
computational mathematics. 

\index{Goldberg, David}
The 1991 paper by Goldberg \cite{WhatEveryComputerScientist} is a general presentation of the floating
point system and its consequences. It begins with background on floating point 
representation and rounding errors, continues with a discussion
of the IEEE floating point standard and concludes with examples of how
computer system builders can better support floating point. The section
1.4, "Cancellation" specifically consider the computation of the roots
of a quadratic equation.

We can also read the numerical experiments performed by Nievergelt in \cite{Nievergelt2003}.

The Numerical Recipes \cite{NumericalRecipes}, chapter 5, section 5.6,
"Quadratic and Cubic Equations" present the elementary theory 
for a floating point implementation of the quadratic and cubic equations.

\index{Kahan, William}
Other references include William Kahan \cite{Kahan2004}.



% Scilab ( http://www.scilab.org/ ) - This file is part of Scilab
% Copyright (C) 2008-2010 - Digiteo - Michael Baudin
%
% This file must be used under the terms of the CeCILL.
% This source file is licensed as described in the file COPYING, which
% you should have received as part of this distribution.  The terms
% are also available at
% http://www.cecill.info/licences/Licence_CeCILL_V2-en.txt

\section{Numerical derivatives}

In this section, we analyze the computation of the numerical derivative of 
a given function.

In the first part, we briefly report the first order forward formula, which 
is based on the Taylor theorem.
We then present the naive algorithm based on these mathematical formulas. 
In the second part, we make some experiments in Scilab and compare our
naive algorithm with the \scifun{derivative} Scilab function.
In the third part, we analyze why and how floating point numbers must be taken 
into account when we compute numerical derivatives.

\subsection{Theory}

The basic result is the Taylor formula with one variable \cite{dixmier}.
Assume that $x\in\RR$ is a given number  and $h\in\RR$ is a given step. Assume 
that $f:\RR\rightarrow\RR$ is a two times continuously differentiable function. 
Therefore,
\begin{eqnarray}
f(x+h) &=& f(x) 
+ h f^\prime(x)
+\frac{h^2}{2} f^{\prime \prime}(x) + \mathcal{O}(h^3).
\end{eqnarray}
We immediately get the forward difference which approximates the first derivate at order 1 
\begin{eqnarray}
f^\prime(x) &=& \frac{f(x+h)  - f(x)}{h} + \frac{h}{2} f^{\prime \prime}(x) + \mathcal{O}(h^2).
\end{eqnarray}

The naive algorithm to compute the numerical derivate of 
a function of one variable is presented in figure \ref{naive-numericalderivative}.

\begin{algorithm}[htbp]
\SetKwInOut{Input}{input}\SetKwInOut{Output}{output}
\Input{$x,h$}
\Output{$f'(x)$}
$f'(x) := (f(x+h)-f(x))/h$\;
\caption{Naive algorithm to compute the numerical derivative of a function of one variable.}
\label{naive-numericalderivative}
\end{algorithm}

\subsection{Experiments}

The following Scilab function \scifun{myfprime} is a straightforward implementation
of the previous algorithm.
\lstset{language=scilabscript}
\begin{lstlisting}
function fp = myfprime(f,x,h)
  fp = (f(x+h) - f(x))/h;
endfunction
\end{lstlisting}

In our experiments, we will compute the derivatives of the 
square function $f(x)=x^2$, which is $f'(x)=2x$.
The following Scilab function \scifun{myfunction} computes the square function.
\lstset{language=scilabscript}
\begin{lstlisting}
function y = myfunction (x)
  y = x*x;
endfunction
\end{lstlisting}

The (naive) idea is that the computed relative error 
is small when the step $h$ is small. Because \emph{small}
is not a priori clear, we take $h= 10^{-16}$
as a "good" candidate for a \emph{small} double.

The \scifun{derivative} function allows to compute the Jacobian and 
the Hessian matrix of a given function.
Moreover, we can use formulas of order 1, 2 or 4. 
The \scifun{derivative} function has been designed by Rainer von Seggern 
and Bruno Pin{\c c}on. The order 1 formula is the forward numerical 
derivative that we have already presented.

\index{\scifun{derivative}}
In the following script, we compare the computed 
relative error produced by our naive method with step
$h=10^{-16}$ and the \scifun{derivative} function with
default optimal step. We compare the two methods for the point $x=1$.
\lstset{language=scilabscript}
\begin{lstlisting}
x = 1.0;
fpref = derivative(myfunction,x,order=1);
e = abs(fpref-2.0)/2.0;
mprintf("Scilab f''=%e, error=%e\n", fpref,e);
h = 1.e-16;
fp = myfprime(myfunction,x,h);
e = abs(fp-2.0)/2.0;
mprintf("Naive f''=%e, h=%e, error=%e\n", fp,h,e);
\end{lstlisting}

The previous script produces the following output.
\begin{lstlisting}
Scilab f'=2.000000e+000, error=7.450581e-009
Naive f'=0.000000e+000, h=1.000000e-016, error=1.000000e+000
\end{lstlisting}

Our naive method seems to be inaccurate and has no 
significant decimal digit. The Scilab function, instead, 
has 9 significant digits.

Since our faith is based on the truth of the mathematical
theory, some deeper experiments must be performed.
We make the following numerical experiment: we take 
the initial step $h=1.0$ and divide $h$ by 10 at each
step of a loop made of 20 iterations.
\lstset{language=scilabscript}
\begin{lstlisting}
x = 1.0;
fpref = derivative(myfunction,x,order=1);
e = abs(fpref-2.0)/2.0;
mprintf("Scilab f''=%e, error=%e\n", fpref,e);
h = 1.0;
for i=1:20
  h=h/10.0;
  fp = myfprime(myfunction,x,h);
  e = abs(fp-2.0)/2.0;
  mprintf("Naive f''=%e, h=%e, error=%e\n", fp,h,e);
end
\end{lstlisting}

The previous script produces the following output.
\begin{lstlisting}
Scilab f'=2.000000e+000, error=7.450581e-009
Naive f'=2.100000e+000, h=1.000000e-001, error=5.000000e-002
Naive f'=2.010000e+000, h=1.000000e-002, error=5.000000e-003
Naive f'=2.001000e+000, h=1.000000e-003, error=5.000000e-004
Naive f'=2.000100e+000, h=1.000000e-004, error=5.000000e-005
Naive f'=2.000010e+000, h=1.000000e-005, error=5.000007e-006
Naive f'=2.000001e+000, h=1.000000e-006, error=4.999622e-007
Naive f'=2.000000e+000, h=1.000000e-007, error=5.054390e-008
Naive f'=2.000000e+000, h=1.000000e-008, error=6.077471e-009
Naive f'=2.000000e+000, h=1.000000e-009, error=8.274037e-008
Naive f'=2.000000e+000, h=1.000000e-010, error=8.274037e-008
Naive f'=2.000000e+000, h=1.000000e-011, error=8.274037e-008
Naive f'=2.000178e+000, h=1.000000e-012, error=8.890058e-005
Naive f'=1.998401e+000, h=1.000000e-013, error=7.992778e-004
Naive f'=1.998401e+000, h=1.000000e-014, error=7.992778e-004
Naive f'=2.220446e+000, h=1.000000e-015, error=1.102230e-001
Naive f'=0.000000e+000, h=1.000000e-016, error=1.000000e+000
Naive f'=0.000000e+000, h=1.000000e-017, error=1.000000e+000
Naive f'=0.000000e+000, h=1.000000e-018, error=1.000000e+000
Naive f'=0.000000e+000, h=1.000000e-019, error=1.000000e+000
Naive f'=0.000000e+000, h=1.000000e-020, error=1.000000e+000
\end{lstlisting}

We see that the relative error begins by decreasing, gets to a minimum 
and then increases. Obviously, the optimum step is approximately $h=10^{-8}$, where the
relative error is approximately $e_r=6.10^{-9}$. 
We should not be surprised to see that Scilab has computed 
a derivative which is near the optimum.

\subsection{Explanations}

In this section, we make reasonable assumptions for the 
expression of the total error and compute the optimal step of a forward
difference formula. We extend our work to the centered two 
points formula.

\subsubsection{Floating point implementation}

The first source of error is obviously the truncation error 
$E_t(h) = h |f^{\prime \prime}(x)|/2$, due to the limited Taylor expansion.

The other source of error is generated by the roundoff errors in the function 
evaluation of the formula $(f(x+h) - f(x))/h$. 
Indeed, the floating point representation of the function value at point $x$ is 
\begin{eqnarray}
fl(f(x)) = (1+e(x))f(x),
\end{eqnarray}
where the relative error $e$ depends on the the current point $x$.
We assume here that the relative error $e$ is bounded by the product of a constant $c>0$
and the machine precision $r$. Furthermore, we assume here that the constant 
$c$ is equal to one.
We may consider other rounding errors sources, such as the error in the 
sum $x+h$, the difference $f(x+h)-f(x)$ or the division $(f(x+h)-f(x))/h$.
But all these rounding errors can be neglected for they are 
not, in general, as large as the roundoff error generated by the function 
evaluation.
Hence, the roundoff error associated with the function evaluation is $E_r(h)=r|f(x)|/h$.

Therefore, the total error associated with the forward finite difference is bounded by 
\begin{eqnarray}
\label{eq-scilabnaive-totalerror}
E(h) = \frac{r|f(x)|}{h} + \frac{h}{2} |f^{\prime \prime}(x)|.
\end{eqnarray}

The error is then the sum of a
term which is a decreasing function of $h$ and a term which an increasing function of $h$.
We consider the problem of finding the step $h$ which minimizes the error $E(h)$.
The total error $E(h)$ is minimized when its first derivative is zero.
The first derivative of the function $E$ is 
\begin{eqnarray}
E^\prime(h) = -\frac{r|f(x)|}{h^2} + \frac{1}{2} |f^{\prime \prime}(x)|.
\end{eqnarray}
The second derivative of $E$ is 
\begin{eqnarray}
E^{\prime\prime}(h) = 2\frac{r|f(x)|}{h^3}.
\end{eqnarray}
If we assume that $f(x)\neq 0$, then the second derivative $E^{\prime\prime}(h)$ is 
strictly positive, since $h>0$ (i.e. we consider only non-zero steps).
Hence, there is only one global solution of the minimization problem.
This first derivative is zero if and only if 
\begin{eqnarray}
-\frac{r|f(x)|}{h^2} + \frac{1}{2} |f^{\prime \prime}(x)| = 0
\end{eqnarray}
Therefore, the optimal step is 
\begin{eqnarray}
\overline{h} = \sqrt{\frac{2r|f(x)|}{|f^{\prime \prime}(x)|}}.
\end{eqnarray}
Let us make the additional assumption 
\begin{eqnarray}
\frac{2|f(x)|}{|f^{\prime \prime}(x)|} \approx 1.
\end{eqnarray}
Then the optimal step is 
\begin{eqnarray}
\overline{h} = \sqrt{r},
\end{eqnarray}
where the error is 
\begin{eqnarray}
E(\overline{h}) = 2 \sqrt{r}.
\end{eqnarray}

With double precision floating point numbers, we have $r=10^{-16}$ and 
we get $\overline{h} = 10^{-8}$ and $E(\overline{h})=2. 10^{-8}$.
Under our assumptions on $f$ and on the form of the total error, this is the 
minimum error which is achievable with a forward difference
numerical derivate.

We can extend the previous method to the first derivate computed by a centered 2 points 
formula. We can prove that 
\begin{eqnarray}
f^\prime(x) &=& \frac{f(x+h) - f(x-h)}{2h} + \frac{h^2}{6} f^{\prime \prime \prime}(x) + \mathcal{O}(h^3).
\end{eqnarray}
We can apply the same method as previously and, under reasonable assumptions
on $f$ and the form of the total error, we get that the 
optimal step is $h = r^{1/3}$, which corresponds to the total error $E=2r^{2/3}$.
With double precision floating point numbers, this corresponds to 
$h \approx 10^{-5}$ and $E\approx 10^{-10}$.

\subsubsection{Robust algorithm}

A more robust algorithm to compute the numerical derivate of 
a function of one variable is presented in figure \ref{robust-numericalderivative}.

\begin{algorithm}[htbp]
$h := \sqrt{r}$\;
$f'(x) := (f(x+h)-f(x))/h$\;
\caption{A more robust algorithm to compute the numerical derivative of a function of one variable.}
\label{robust-numericalderivative}
\end{algorithm}

\subsection{One more step}

In this section, we analyze the behavior of the \scifun{derivative} function 
when the point $x$ is either large in magnitude, 
small or close to zero. 
We compare these results with the \scifun{numdiff} function,
which does not use the same step strategy. As we are going 
to see, both functions performs the same when $x$ is near 1, but 
performs very differently when x is large or small.

The \scifun{derivative} function uses the optimal step 
based on the theory we have presented. But the optimal step does not 
solve all the problems that may occur in practice, as we are going to 
see. 

See for example the following Scilab session, where we compute the 
numerical derivative of $f(x)=x^2$ for $x=10^{-100}$. The 
expected result is $f'(x) = 2. \times 10^{-100}$.
\begin{lstlisting}
-->fp = derivative(myfunction,1.e-100,order=1)
 fp  =
    0.0000000149011611938477  
-->fe=2.e-100
 fe  =
    2.000000000000000040-100  
-->e = abs(fp-fe)/fe
 e  =
    7.450580596923828243D+91  
\end{lstlisting}

The result does not have any significant digit.

The explanation is that the step is $h = \sqrt{r}\approx 10^{-8}$.
Then, the point $x+h$ is computed as $10^{-100} + 10^{-8}$ which is 
represented by a floating point number which is close to $10^{-8}$, because the 
term $10^{-100}$ is much smaller than $10^{-8}$. 
Then we evaluate the function, which leads to $f(x+h)=f(10^{-8}) = 10^{-16}$. The result of the 
computation is therefore $(f(x+h) - f(x))/h = (10^{-16} + 10^{-200})/10^{-8} \approx 10^{-8}$.

That experiment shows that the \scifun{derivative} function uses a 
poor default step $h$ when $x$ is very small.

To improve the accuracy of the computation, we can take the control of the 
step $h$. A reasonable solution is to use $h=\sqrt{r}|x|$ so that the 
step is scaled depending on $x$. 
The following script illustrates than method, which produces 
results with 8 significant digits.
\begin{lstlisting}
-->fp = derivative(myfunction,1.e-100,order=1,h=sqrt(%eps)*1.e-100)
 fp  =
    2.000000013099139394-100  
-->fe=2.e-100
 fe  =
    2.000000000000000040-100  
-->e = abs(fp-fe)/fe
 e  =
    0.0000000065495696770794  
\end{lstlisting}

But when $x$ is exactly zero, the step $h=\sqrt{r}|x|$ cannot work, because 
it would produce the step $h=0$, which would generate a division by zero
exception. In that case, the step $h=\sqrt{r}$ provides a sufficiently 
good accuracy.

Another function is available in Scilab to compute the 
numerical derivatives of a given function, that is \scifun{numdiff}.
The \scifun{numdiff} function uses the step 
\begin{eqnarray}
h=\sqrt{r}(1+10^{-3}|x|).
\end{eqnarray}
In the following paragraphs, we analyze why this formula 
has been chosen. As we are going to check experimentally, this step
formula performs better than \scifun{derivative} when $x$ is 
large, but performs equally bad when $x$ is small.

\index{\scifun{numdiff}}
As we can see the following session, the behavior is approximately 
the same when the value of $x$ is 1.
\begin{lstlisting}
-->fp = numdiff(myfunction,1.0)
 fp  =
    2.0000000189353417390237  
-->fe=2.0
 fe  =
    2.  
-->e = abs(fp-fe)/fe
 e  =
    9.468D-09  
\end{lstlisting}

The accuracy is slightly decreased with respect to the optimal
value 7.450581e-009 which was produced by the \scifun{derivative} function. 
But the number of significant digits is approximately the same, i.e. 9 digits.

The goal of the step used by the \scifun{numdiff} function is to produce good 
accuracy when the value of $x$ is large. In this case, the \scifun{numdiff} function 
produces accurate results, while the \scifun{derivative} function performs poorly.

In the following session, we compute the numerical derivative of the function $f(x)=x^2$
at the point $x=10^{10}$. The expected result is $f'(x)=2.10^{10}$.
\begin{lstlisting}
-->numdiff(myfunction,1.e10)
  ans  =
    2.000D+10  
-->derivative(myfunction,1.e10,order=1)
 ans  =
    0.  
\end{lstlisting}
We see that the \scifun{numdiff} function produces an accurate result while the 
\scifun{derivative} function produces a result which has no significant digit.

The behavior of the two functions when $x$ is close to zero is the same, i.e. both functions 
produce wrong results. Indeed, when we use the \scifun{derivative} function, the 
step $h=\sqrt{r}$ is too large so that the point $x$ is neglected against the step $h$. 
On the other hand, we we use the \scifun{numdiff} function, the step $h=\sqrt{r}(1+10^{-3}|x|)$
is approximated by $h=\sqrt{r}$ so that it produces the same 
results as the \scifun{derivative} function.

\subsection{References}

A reference for numerical derivatives 
is \cite{AbramowitzStegun1972}, chapter 25. "Numerical Interpolation, 
Differentiation and Integration" (p. 875).
The webpage \cite{schimdtnd} and the book \cite{NumericalRecipes} give
results about the rounding errors.

In order to solve this issue generated by the magnitude of $x$, 
more complex methods should be used.
Moreover, we did not give the solution of other sources of rounding errors.
Indeed, the step $h=\sqrt{r}$ was computed based on assumptions on the 
rounding error of the function evaluations, where we consider that the 
constant $c$ is equal to one. This assumption is satisfied only
in the ideal case. Furthermore, we make the assumption that 
the factor $\frac{2|f(x)|}{|f^{\prime \prime}(x)|}$ is close to one.
This assumption is far from being achieved in practical situations,
where the function value and its second derivative can vary greatly 
in magnitude.

Several authors attempted to solve the problems associated with 
numerical derivatives.
A non-exhaustive list of references includes \cite{KelleyNewtonMethod,DumontetVignes1977,1979SteplemanWinarsky,Gill81MurrayWright}.




% Scilab ( http://www.scilab.org/ ) - This file is part of Scilab
% Copyright (C) 2008-2010 - Digiteo - Michael Baudin
%
% This file must be used under the terms of the CeCILL.
% This source file is licensed as described in the file COPYING, which
% you should have received as part of this distribution.  The terms
% are also available at
% http://www.cecill.info/licences/Licence_CeCILL_V2-en.txt

\section{Complex division}

In this section, we analyze the problem of the complex division in Scilab.
We especially detail the difference between the mathematical, straightforward
formula and the floating point implementation. In the first part, we briefly report 
the formulas which allow to 
compute the real and imaginary parts of the division of two complex numbers.
We then present the naive algorithm based on these mathematical formulas. 
In the second part, we make some experiments in Scilab and compare our
naive algorithm with Scilab's division operator.
In the third part, we analyze 
why and how floating point numbers must be taken into account when the 
implementation of such division is required.

\subsection{Theory}

Assume that $a,b,c$ and $d$ are four real numbers.
Consider the two complex numbers $a + ib$ and $c + id$, where $i$ is 
the imaginary number which satisfies $i^2=-1$. Assume that $c^2 + d^2$ is non zero.
We are interested in the complex number $e+fi = \frac{a + ib}{c + id}$ where $e$ and $f$ are real
numbers.
The formula which allows to compute the real and imaginary parts 
of the division of these two complex numbers is 
\begin{eqnarray}
\label{compdiv-eq-defcomplexdiv}
\frac{a + ib}{c + id} = \frac{ac + bd}{c^2 + d^2} + i \frac{bc - ad}{c^2 + d^2} .
\end{eqnarray}
So that the real and imaginary parts $e$ and $f$ of the complex number are
\begin{eqnarray}
\label{compdiv-eq-e}
e &=& \frac{ac + bd}{c^2 + d^2}, \\
\label{compdiv-eq-f}
f &=& \frac{bc - ad}{c^2 + d^2}.
\end{eqnarray}

The naive algorithm for the computation of the complex division
is presented in figure \ref{naive-complexdivision}.

\begin{algorithm}[htbp]
\SetKwInOut{Input}{input}\SetKwInOut{Output}{output}
\Input{$a,b,c,d$}
\Output{$e,f$}
$den := c^2 + d^2$\;
$e := (ac + bd)/ den$\;
$f := (bc - ad)/ den$\;
\caption{Naive algorithm to compute the complex division. The algorithm takes as input the 
real and imaginary parts $a,b,c,d$ of the two complex numbers and returns 
$e$ and $f$, the real and imaginary parts of the division.}
\label{naive-complexdivision}
\end{algorithm}

\subsection{Experiments}

The following Scilab function \scifun{naive} is a straightforward implementation
of the previous formulas. It takes as input the complex numbers $a$ and $b$, 
represented by their real and imaginary parts \scivar{a}, \scivar{b}, \scivar{c} and 
\scivar{d}. The function \scifun{naive} returns the complex number represented 
by its real and imaginary parts \scivar{e} and \scivar{f}.
\lstset{language=scilabscript}
\begin{lstlisting}
function [e,f] = naive (a , b , c , d )
  den = c * c + d * d;
  e = (a * c + b * d) / den;
  f = (b * c - a * d) / den;
endfunction
\end{lstlisting}

Consider the complex division
\begin{eqnarray}
\label{eq-cd-firstest}
\frac{1 + i2}{3 + i4} = \frac{11}{25}+i \frac{2}{25}= 0.44 +i 0.08 .
\end{eqnarray}
We check our result with Wolfram Alpha\cite{WWWWolframAlpha}, with 
the input "(1+i*2)/(3+i*4)".
In the following script, we check that there is no obvious bug 
in the naive implementation.
\lstset{language=scilabscript}
\begin{lstlisting}
-->  [e f] = naive ( 1.0 , 2.0 , 3.0 , 4.0 )
 f  =
    0.08  
 e  =
    0.44  
-->  (1.0 + %i * 2.0 )/(3.0 + %i * 4.0 )
 ans  =
    0.44 + 0.08i  
\end{lstlisting}
The results of the \scifun{naive} function and the division operator are the same,
which makes us confident that our implementation is correct.

Now that we are confident, we make the following numerical experiment involving 
a large number. Consider the complex division 
\begin{eqnarray}
\label{eq-cd-10307}
\frac{1 + i}{1 + i 10^{307}  } \approx 1.0000000000000000\cdot 10^{-307} - i 1.0000000000000000\cdot 10^{-307} ,
\end{eqnarray}
which is accurate to the displayed digits. 
We check our result with Wolfram Alpha\cite{WWWWolframAlpha}, with 
the input "(1 + i)/(1 + i * 10$\hat{\;}$307)".
In fact, there are more that 300 zeros following the leading 
1, so that the previous approximation is very accurate.
The following Scilab session compares the naive implementation and Scilab's division operator.
\lstset{language=scilabscript}
\begin{lstlisting}
-->  [e f] = naive ( 1.0 , 1.0 , 1.0 , 1.e307 )
 f  =
    0.  
 e  =
    0.  
-->  (1.0 + %i * 1.0)/(1.0 + %i * 1.e307)
 ans  =
    1.000-307 - 1.000-307i  
\end{lstlisting}
In the previous case, the naive implementation does not produce any correct digit!

The last test involves small numbers in the denominator of the complex fraction.
Consider the complex division 
\begin{eqnarray}
\label{eq-cd-thirdtest}
\frac{1 + i}{10^{-307} +  i 10^{-307} }= \frac{1 + i}{10^{-307}(1 +  i)} = 10^{307}.
\end{eqnarray}
In the following session, the first statement \scifun{ieee(2)} configures the 
IEEE system so that Inf and Nan numbers are generated instead 
of Scilab error messages. 
\lstset{language=scilabscript}
\begin{lstlisting}
-->ieee(2);
-->[e f] = naive ( 1.0 , 1.0 , 1.e-307 , 1.e-307 )
 f  =
   Nan  
 e  =
   Inf  
-->(1.0 + %i * 1.0)/(1.e-307 + %i * 1.e-307)
 ans  =
    1.000+307  
\end{lstlisting}
We see that the naive implementation generates the IEEE numbers Nan and Inf, while the 
division operator produces the correct result.

\subsection{Explanations}

In this section, we analyze the reason why the naive implementation
of the complex division leads to inaccurate results.
In the first section, we perform algebraic computations 
and shows the problems of the naive formulas.
In the second section, we present the Smith's method.

\subsubsection{Algebraic computations}

In this section, we analyze the results produced by the second and third tests in the 
previous numerical experiments. We show that the intermediate numbers which appear
are not representable as double precision floating point numbers.

Let us analyze the second complex division \ref{eq-cd-10307}.
We are going to see that this division is associated with 
an IEEE overflow.
We have $a=1$, $b=1$, $c=1$ and $d=10^{307}$.
By the equations \ref{compdiv-eq-e} and \ref{compdiv-eq-f}, we have 
\begin{eqnarray}
den &=& c^2 + d^2 = 1^2 + (10^{307})^2 \\
  &=& 1 + 10^{614} \approx 10^{614}, \\
e &=& (ac + bd)/ den = (1*1 + 1*10^{307})/10^{614}, \\
  &\approx& 10^{307}/10^{614}  \approx 10^{-307},\\
f &=& (bc - ad)/ den = (1*1 - 1*10^{307})/10^{614} \\
  &\approx& -10^{307}/10^{614} \approx -10^{-307}.
\end{eqnarray}
We see that both the input numbers $a,b,c,d$ are representable 
and the output numbers $e=10^{-307}$ and $f=-10^{-307}$ are representable as double precision
floating point numbers.
We now focus on the floating point representation of the intermediate expressions.
We have 
\begin{eqnarray}
fl(den) = fl(10^{614}) = Inf,
\end{eqnarray}
because $10^{614}$ is not representable as a double precision number. 
Indeed, the largest positive double is $10^{308}$.
The IEEE Inf floating point number stands for Infinity and is associated with an overflow. 
The Inf floating point number is associated with an algebra which defines that $1/Inf = 0$. 
This is consistent with mathematical limit of the function $1/x$ when $x\rightarrow \infty$.
Then, the $e$ and $f$ terms are computed as 
\begin{eqnarray}
fl(e) &=& fl((ac + bd)/ den) = fl((1*1 + 1*10^{307})/Inf) = fl(10^{307}/Inf) = 0,\\
fl(f) &=& fl((bc - ad)/ den) = fl((1*1 - 1*10^{307})/Inf) = fl(-10^{307}/Inf) = 0.
\end{eqnarray}
Hence, the result is computed without any significant digit,
even though both the input and the output numbers are all representable as double precision
floating point numbers.

Let us analyze the second complex division \ref{eq-cd-thirdtest}.
We are going to see that this division is associated with 
an IEEE underflow.
We have $a=1$, $b=1$, $c=10^{-307}$ and $d=10^{-307}$.
We now use the equations \ref{compdiv-eq-e} and \ref{compdiv-eq-f}, which leads to:
\begin{eqnarray}
den &=& c^2 + d^2 = (10^{-307})^2 + (10^{-307})^2 \\
  &=& 10^{-614} + 10^{-614} = 2 . 10^{-616}, \\
e &=& (ac + bd)/ den = (1*10^{-307} + 1*10^{-307})/(2 . 10^{-614}) \\
 &=&  (2 . 10^{-307})/(2 . 10^{-614}) = 10^{307}, \\
f &=& (bc - ad)/ den = (1*10^{-307} - 1*10^{-307})/(2 . 10^{-614}) \\
  &=& 0/10^{-614} = 0.
\end{eqnarray}
With double precision floating point numbers, the computation is not performed this way.
The positive terms which are smaller than $10^{-324}$ are too small to be representable 
in double precision and are represented by 0 so that an underflow occurs.
This leads to
\begin{eqnarray}
fl(den) &=& fl(c^2 + d^2) = fl(10^{-614} + 10^{-614}) \\
&=& 0, \\
fl(e) &=& fl((ac + bd)/ den) = fl((1*10^{-307} + 1*10^{-307})/(2 . 10^{-614}))  \\
&=& fl( 2. 10^{-307}/0 ) = Inf, \\
fl(f) &=& fl((bc - ad)/ den) = fl((1*10^{-307} - 1*10^{-307})/0)  \\
&=& fl(0/0) = NaN.
\end{eqnarray}

The two previous examples shows that, even if both the input and output numbers are 
representable as floating point numbers, the intermediate expressions 
may generate numbers which may not be representable as floating point 
numbers. Hence, a naive implementation can lead to inaccurate results.
In the next section, we present a method which allows to cure most 
problems generated by the complex division.

\subsubsection{Smith's method}

In this section, we analyze Smith's method, which allows to produce an accurate 
division of two complex numbers. We present the detailed steps of this modified 
algorithm in the particular cases that we have presented.

\index{Smith, Robert}
In Scilab, the algorithm which allows to perform the complex 
division is done by the the \emph{wwdiv} routine, which implements  
Smith's method \cite{Smith1962}.
This implementation is due to Bruno Pin{\c c}on.
Smith's algorithm is based on normalization, which allow to 
perform the complex division even if the input terms are large or small. 

The starting point of the method is the mathematical definition \ref{compdiv-eq-defcomplexdiv},
which is reproduced here for simplicity
\begin{eqnarray}
\label{compdiv-eq-defcomplexdiv2}
\frac{a + ib}{c + id} = e+if = \frac{ac + bd}{c^2 + d^2} + i \frac{bc - ad}{c^2 + d^2}.
\end{eqnarray}

Smith's method is based on the rewriting of this formula in 
two different, but mathematically equivalent, formulas. 
We have seen that the term $c^2 + d^2$ may generate overflows or underflows.
This is caused by intermediate expressions which magnitudes are larger than necessary.
The previous numerical experiments suggest that, provided that we 
had simplified the calculation, the intermediate expressions 
would not have been unnecessary large. 

Consider the term $e=\frac{ac + bd}{c^2 + d^2}$ in the equation \ref{compdiv-eq-defcomplexdiv2}
and assume that $c\neq 0$ and $d\neq0$.
Let us assume that $c$ is large in magnitude with respect to $d$, i.e. $|d|\ll|c|$. 
This implies $d/c \leq 1$. We see that the denominator $c^2 + d^2$ 
squares the number $c$, which also appears in the numerator. Therefore, 
we multiply both the numerator and the denominator by $1/c$. If we  
express $e$ as $\frac{a + b(d/c)}{c + d (d/c)}$, which is 
mathematically equivalent, we see that there is no more squaring of $c$.
Hence, overflows are less likely to occur in the denominator, since $|d (d/c)|= |d| |d/c| \leq |d|$.
That is, there is no growth in the magnitude of the terms involved in the 
computation of the product $d(d/c)$.
Similarly, overflows are less likely to occur in the numerator, since 
$|b(d/c)|= |b| |d/c| \leq |b|$. In the opposite case where $d$ is large with 
respect to $c$, i.e. $d\gg c$, we could divide the numerator and the denominator by $1/d$.
This leads to the formulas
\begin{eqnarray}
\frac{a + ib}{c + id} 
&=& \frac{a + b(d/c)}{c + d(d/c)} + i \frac{b - a(d/c)}{c + d(d/c)}, \\
&=& \frac{a(c/d) + b}{c(c/d) + d} + i \frac{b(c/d) - a}{c(c/d) + d}. 
\end{eqnarray}

The previous equations can be simplified as 
\begin{eqnarray}
\frac{a + ib}{c + id} 
&=& \frac{a + br}{c + dr} + i \frac{b - ar}{c + dr} , \qquad r = d/c, \textrm{ if } c\geq d,\\
&=& \frac{ar + b}{cr + d} + i \frac{br - a}{cr + d}, \qquad r = c/d, \textrm{ if } d\geq c.
\end{eqnarray}

The following \scifun{smith} function implements 
Smith's method in the Scilab language.
\lstset{language=scilabscript}
\begin{lstlisting}
function [e,f] = smith ( a , b , c , d )
  if ( abs(d) <= abs(c) ) then
    r = d/c;
    den = c + d * r;
    e = (a + b * r) / den;
    f = (b - a * r) / den;
  else
    r = c/d;
    den = c * r + d;
    e = (a * r + b) / den;
    f = (b * r - a) / den;
  end
endfunction
\end{lstlisting}

We now check that Smith's method performs very well for the difficult complex 
division that we met earlier in this chapter.

Let us analyze the second complex division \ref{eq-cd-10307}.
We have $a=1$, $b=1$, $c=1$ and $d=10^{307}$.
For this division, Smith's method is the following.
\begin{lstlisting}
if ( |1.e307| <= |1| ) > test false
else
  r = c/d = 1 / 1.e307 = 1.e-307
  den  = c * r + d = 1 * 1.e-307 + 1.e307 = 1.e307
  e = (a * r + b)/den = (1 * 1.e-307 + 1) / 1.e307 = 1 / 1.e307 
    = 1.e-307
  f = (b * r - a)/den = (1 * 1.e-307 - 1) / 1.e307 = -1 / 1.e307 
    = -1.e-308
\end{lstlisting}
We see that, while the naive division generated an overflow, Smith's 
method produces the correct result.

Let us analyze the second complex division \ref{eq-cd-thirdtest}.
We have $a=1$, $b=1$, $c=10^{-307}$ and $d=10^{-307}$.
\begin{lstlisting}
if ( |1.e-307| <= |1.e-307| ) > test true
  r = d/c = 1.e-307 / 1.e-307 = 1
  den  = c + d * r = 1.e-307 +  1e-307 * 1 = 2.e-307
  e = (a + b * r) / den = (1 + 1 * 1) / 2.e-307 = 2/2.e-307 
    = 1.e307
  f = (b - a * r) / den = (1 - 1 * 1) / 2.e-307 
    = 0
\end{lstlisting}
We see that, while the naive division generated an underflow, Smith's 
method produces the correct result.

Now that we have designed a more robust algorithm, we are interested in 
testing Smith's method on a more difficult case.

\subsection{One more step}

In this section, we show the limitations of Smith's method
and present an example where Smith's method does not perform as 
expected.

The following example is inspired by an example by Stewart's in \cite{214414}. 
While Stewart gives an example based on a machine with an exponent range 
$\pm 99$, we consider an example which is based on Scilab's doubles. 
Consider the complex division
\begin{eqnarray}
\frac{10^{307} +  i 10^{-307}}{10^{204} +   i 10^{-204}} 
\approx 1.000000000000000 \cdot 10^{103} -  i 1.000000000000000 \cdot 10^{-305},
\end{eqnarray}
which is accurate to the displayed digits. In fact, there are more that 100 zeros following the leading 
1, so that the previous approximation is very accurate.
The following Scilab session compares the naive implementation, Smith's method
and Scilab's division operator.
The session is performed with Scilab v5.2.0 under a 32 bits Windows
using a Intel Xeon processor.
\lstset{language=scilabscript}
\begin{lstlisting}
-->[e f] = naive ( 1.e307 , 1.e-307 , 1.e204 , 1.e-204 )
 f  =
    0.  
 e  =
   Nan  
-->[e f] = smith ( 1.e307 , 1.e-307 , 1.e204 , 1.e-204 )
 f  =
    0.  
 e  =
    1.000+103  
-->(1.e307 + %i * 1.e-307)/(1.e204 + %i * 1.e-204)
 ans  =
    1.000+103 - 1.000-305i  
\end{lstlisting}
In the previous case, the naive implementation does not produce any correct digit, as 
expected. Smith's method, produces a correct real part, but an inaccurate imaginary 
part. Once again, Scilab's division operator provides the correct 
answer.

We first check why the naive implementation is not accurate in this case.
We have $a=10^{307}$, $b=10^{-307}$, $c=10^{204}$ and $d=10^{-204}$.
Indeed, the naive implementation performs the following steps.
\begin{lstlisting}
  den = c * c + d * d = 1.e204 * 1.e204 + 1.e-204 * 1.e-204 
      = Inf
  e = (a * c + b * d) / den 
    = (1.e307 * 1.e204 + 1.e-307 * 1.e-204 ) / Inf = Inf / Inf 
    = Nan
  f = (b * c - a * d) / den 
    = (1.e-307 * 1.e204 - 1.e307 * 1.e-204) / Inf = -1.e103 / Inf 
    = 0
\end{lstlisting}
We see that the denominator \scivar{den} is in overflow, which 
makes \scivar{e} to be computed as \scivar{Nan} and \scivar{f} 
to be computed as \scivar{0}.

Second, we check that Smith's formula is not accurate in this 
case. Indeed, it performs the following steps.
\begin{lstlisting}
if ( abs(d) = 1.e-204 <= abs(c) = 1.e204 ) > test true
  r = d/c = 1.e-204 / 1.e204 = 0
  den  = c + d * r = 1.e204 +  0 * 1.e-204 = 1.e204
  e = (a + b * r) / den = (1.e307 + 1.e-307 * 0) / 1e204 
    = 1.e307 / 1.e204 = 1.e103
  f = (b - a * r) / den = (1.e-307 - 1.e307 * 0) / 1e204 
    = 1.e-307 / 1.e204 = 0
\end{lstlisting}
We see that the variable \scivar{r} is in underflow, so that it is 
represented by zero. This simplifies the denominator \scivar{den},
but this variable is still correctly computed, because it is dominated 
the term \scivar{c}. The real part \scivar{e} is still accurate, because,
once again, the computation is dominated by the term \scivar{a}.
The imaginary part \scivar{f} is wrong, because this term should be 
dominated by the term \scivar{a*r}. Since \scivar{r} is in underflow, it 
is represented by zero, which completely changes the result of the 
expression \scivar{b-a*r}, which is now equal to \scivar{b}.
Therefore, the result is equal to \scivar{1.e-307 / 1.e204}, which 
underflows to zero. 

Finally, we analyze why Scilab's division operator performs 
accurately in this case. Indeed, the formula used by Scilab 
is based on Smith's method and we proved that this method 
fails in this case, when we use double floating point numbers. 
Therefore, we experienced here an unexpected high accuracy.

We performed this particular complex division over several common 
computing systems such as various versions of Scilab, Octave, 
Matlab and FreeMat on various operating systems. The results are presented
in figure \ref{fig-compdiv-weird}. Notice that, on Matlab, 
Octave and FreeMat, the syntax is different and we 
used the expression \scivar{(1.e307 + i * 1.e-307)/(1.e204 + i * 1.e-204)}.

\begin{figure}
\begin{center}
\begin{tabular}{|l|l|l|}
\hline
Scilab v5.2.0 release & Windows 32 bits & 1.000+103 - 1.000-305i \\
Scilab v5.2.0 release & Windows 64 bits & 1.000+103 \\
Scilab v5.2.0 debug   & Windows 32 bits & 1.000+103 \\
Scilab v5.1.1 release & Windows 32 bits & 1.000+103 \\
Scilab v4.1.2 release & Windows 32 bits & 1.000+103 \\
Scilab v5.2.0 release & Linux 32 bits   & 1.000+103 - 1.000-305i \\
Scilab v5.1.1 release & Linux 32 bits   & 1.000+103 - 1.000-305i \\
Octave v3.0.3         & Windows 32 bits & 1.0000e+103 \\
Matlab 2008           & Windows 32 bits & 1.0000e+103 -1.0000e-305i \\
Matlab 2008           & Windows 64 bits & 1.0000e+103 \\
FreeMat v3.6          & Windows 32 bits & 1.0000e+103 -1.0000e-305i \\
\hline
\end{tabular}
\end{center}
\caption{Result of the complex division \scivar{(1.e307 + \%i * 1.e-307)/(1.e204 + \%i * 1.e-204)} on various 
softwares and operating systems.}
\label{fig-compdiv-weird}
\end{figure}

The reason of the discrepancies of the results is the following \cite{MullerEtAl2010,Monniaux2008}. 
The processor being used may offer an internal precision that is wider than the 
precision of the variables of a program. 
Indeed, processors of the IA32 architecture (Intel 386, 486, Pentium etc. and 
compatibles) feature a floating-point unit often known as "x87".
This unit has 80-bit registers in "double extended" format with a 64-bit mantissa and
a 15-bit exponent. The most usual way of generating code for the IA32 is to hold temporaries -
and, in optimized code, program variables - in the x87 registers.
Hence, the final result of the computations depend on how the compiler allocates registers.
Since the double extended format of the x87 unit uses 15 bits for the exponent, 
it can store floating point numbers associated with binary exponents from 
$2^{-16382}\approx 10^{-4932}$ up to $2^{16383}\approx 10^{4931}$,
which is much larger than the exponents from the 64-bits double precision 
floating point numbers (ranging from $2^{-1022}\approx 10^{-308}$ up to $2^{1023}\approx 10^{307})$.
Therefore, the computations performed with the x87 unit are less likely to 
generate underflows and overflows. On the other hand, SSE2 extensions introduced 
one 128-bit packed floating-point data type. This 128-bit data type consists of
two IEEE 64-bit double-precision floating-point values packed into a double
quadword.

Depending on the compilers options used to generate the binary,
the result may use either the x87 unit (with 80-bits registers) or the 
SSE unit. Under Windows 32 bits, Scilab v5.2.0 is compiled with the "/arch:IA32" 
option \cite{CordenKreitzerIntel2009}, which allows Scilab to run on 
older Pentium computers that does not support SSE2. In this situation,
Scilab may use the x87 unit. Under Windows 64 bits, Scilab uses the SSE2 unit so that the result 
is based on double precision floating point numbers only.
Under Linux, Scilab is compiled with gcc \cite{GCCManual2008}, 
where the behavior is driven by the -mfpmath option. The default value of this option for i386 machines 
is to use the 387 floating point co-processor while, for x86\_64 machines, the default 
is to use the SSE instruction set.

\subsection{References}

\index{Smith, Robert}
The 1962 paper by R. Smith \cite{Smith1962} describes the algorithm which is used in 
Scilab. 

\index{Goldberg, David}
Goldberg introduces in \cite{WhatEveryComputerScientist} many 
of the subjects presented in this document, including the problem of the 
complex division. 

An analysis of Hough, cited by Coonen \cite{1667289} and Stewart \cite{214414} 
shows that when the algorithm works, it returns a computed value $\overline{z}$
satisfying
\begin{eqnarray}
|\overline{z} - z| <= eps |z|,
\end{eqnarray}
where $z$ is the exact complex division result and $\epsilon$ is of the same order of magnitude 
as the rounding unit for the arithmetic in question. 

The limits of Smith's method have been analyzed by Stewart's in \cite{214414}.
The paper separates the relative error of the complex numbers and the relative
error made on real and imaginary parts. 
Stewart's algorithm is based on a theorem which states that if $x_1 \ldots x_n$
are $n$ floating point representable numbers, and if their product is also 
a representable floating point number, then the product $\min_{i=1,n}(x_i) . \max_{i=1,n}(x_i)$
is also representable. The algorithm uses that theorem to perform a 
correct computation.

\index{Kahan, William}
\index{Stewart, G. W.}
Stewart's algorithm is superseded by the one by Li et Al \cite{567808}, but 
also by Kahan's \cite{KAHAN1987}, which, from  \cite{1039814}, is the one implemented
in the C99 standard.

\index{Knuth, Donald E.}
Knuth presents in \cite{artcomputerKnuthVol2}
the Smith's method in section 4.2.1, as exercize 16. Knuth gives also 
references \cite{Wynn:1962:AAP} and \cite{DBLP:journals/cacm/Friedland67}.
The 1967 paper by Friedland \cite{DBLP:journals/cacm/Friedland67} describes 
two algorithms to compute the absolute value of a complex number 
$|x+iy| = \sqrt{x^2+y^2}$ and the square root of a 
complex number $\sqrt{x+iy}$.

\index{Muller, Jean-Michel}
Issues related to the use of extended double precision floating point 
numbers are analyzed by Muller et al. in \cite{MullerEtAl2010}. 
In the section 3 of part I, "Floating point formats an Environment", 
the authors analyze the "double rounding" problem which occurs when an 
internal precision is wider than the precision of the variables of a 
program. The typical example is the double-extended format available 
on Intel platforms. Muller et al. show different examples, where the 
result depends on the compiler options and the platform, including an example extracted 
from a paper by Monniaux \cite{Monniaux2008}.

\index{Corden, Martyn}
\index{Kreitzer, David}
Corden and Kreitzer analyse in \cite{CordenKreitzerIntel2009} the 
effect of the Intel compiler floating point options on the numerical results.
The paper focuses on the reproductibility issues which are associated 
with floating point computations. The options which allow to be compliant 
with the IEEE standards for C++ and Fortran are presented.
The effects of optimization options is considered with respect to 
speed and the safety of the transformations that may be done on the 
source code.

The "Intel 64 and IA-32 Architectures Software Developer's Manual. Volume 1: Basic Architecture" 
\cite{Intel64IA32Architecture} is part of a set of documents that describes the architecture
and programming environment of Intel 64 and IA-32 architecture processors. The chapter 8, "Programming 
with the x87 environment" presents the registers and the 
instruction set for this unit. The section 8.1.2, "x87 FPU Data Registers" focuses 
on the floating point registers, which are based on 80-bits and implements 
the double extended-precision floating-point format. 
The chapter 10, "Programming with Streaming SIMD Extensions (SSE)" introduces the 
extensions which were introduced into the IA-32 architecture in
the Pentium III processor family. The chapter 11 introduces the SSE2 extensions.

\index{Monniaux, David}
In \cite{Monniaux2008}, David Monniaux presents issues related to the 
analysis of floating point programs. He emphasizes the difficulty
of defining the semantics of common implementation of floating point numbers,
depending on choices made by the compiler. He gives concrete examples of problems that 
can appear and solutions.



% Copyright (C) 2008-2010 - Consortium Scilab - Digiteo - Michael Baudin
%
% This file must be used under the terms of the 
% Creative Commons Attribution-ShareAlike 3.0 Unported License :
% http://creativecommons.org/licenses/by-sa/3.0/

\section{Conclusion}

We have presented several cases where the mathematically perfect 
algorithm (i.e. without obvious bugs) does not produce accurate results 
with the computer in particular situations.
Many Scilab algorithms take floating point values as inputs,
and return floating point values as output. We have presented situations 
where the intermediate calculations involve terms which are 
not representable as floating point values. We have also presented 
examples where cancellation occurs so that the rounding errors dominate 
the result. We have analyzed specific algorithms which can be used to 
cure some of the problems. 

Most algorithms provided by Scilab are designed specifically to take into 
account for floating point numbers issues. The result is a collection of 
robust algorithms which, most of the time, exceed the user's needs.

Still, it may happen that the algorithm used by Scilab is not accurate enough,
so that floating point issues may occur in particular cases. We cannot 
pretend that Scilab always use the best algorithm. In fact, we have 
given in this document practical (but extreme) examples where the 
algorithm used by Scilab is not accurate. In this situation, an 
interesting point is that Scilab is open-source, so that anyone who want 
can inspect the source code, analyze the algorithm and point out the 
problems of this algorithm.

That article does not aim at discouraging from 
using floating point numbers or implementing our own algorithms.
Instead, the goal of this document is to give examples where 
some specific work is to do when we translate the 
mathematical material into a computational algorithm based on floating 
point numbers. Indeed, accuracy can be obtained with floating point numbers, 
provided that we are less \emph{naive}, use the appropriate theory and algorithms, and perform 
the computations with tested softwares.


% Copyright (C) 2008-2010 - Consortium Scilab - Digiteo - Michael Baudin
%
% This file must be used under the terms of the 
% Creative Commons Attribution-ShareAlike 3.0 Unported License :
% http://creativecommons.org/licenses/by-sa/3.0/

\chapter{Acknowledgments}

I would like to thank Vincent Couvert, 
the team manager for Scilab releases, for his support 
during the development of this software. I would like to thank 
Serge Steer, INRIA researcher, for his comments and the discussions 
on this subject. Professor Han, Associate Professor of Mathematics in the 
University of Michigan-Flint University was kind enough to send me a copy
of his Phd and I would like to thank him for that.
My colleagues Allan Cornet and Yann Collette helped me in many 
steps in the long process from the initial idea to the final 
release of the tool and I would like to thank them for their
their time.



% Scilab ( http://www.scilab.org/ ) - This file is part of Scilab
% Copyright (C) 2008-2010 - Digiteo - Michael Baudin
%
% This file must be used under the terms of the CeCILL.
% This source file is licensed as described in the file COPYING, which
% you should have received as part of this distribution.  The terms
% are also available at
% http://www.cecill.info/licences/Licence_CeCILL_V2-en.txt


\section{Appendix}

In this section, we analyze the examples given in the introduction of this 
article. In the first section, we analyze how the real number $0.1$ is represented by a double precision
floating point number, which leads to a rounding error. In the second 
section, we analyze how the computation of $\sin(\pi)$ is performed.
In the final section, we make an experiment which shows that $\sin(2^{10i} \pi)$ can 
be arbitrary far from zero when we compute it with double precision floating point 
numbers.

\subsection{Why $0.1$ is rounded}

In this section, we present a brief explanation for the 
following Scilab session.
\begin{lstlisting}
-->format(25)
-->x1=0.1
 x1  =
    0.1000000000000000055511  
-->x2 = 1.0-0.9
 x2  =
    0.0999999999999999777955  
-->x1==x2
 ans  =
  F  
\end{lstlisting}

We see that the real decimal number 0.1 is displayed as $0.100000000000000005$.
In fact, only the 17 first digits after the decimal point are 
significant : the last digits are a consequence of the approximate conversion 
from the internal binary double number to the decimal number.

In order to understand what happens, we must decompose the 
floating point number into its binary components. 
The IEEE double precision floating point numbers used by Scilab 
are associated with a radix (or basis) $\beta=2$, 
a precision $p=53$, a minimum exponent $e_{min}=-1023$
and a maximum exponent $e_{max}=1024$.
Any floating point number $x$ is represented as 
\begin{eqnarray}
fl(x) = M \cdot \beta^{e-p+1},
\end{eqnarray}
where 
\begin{itemize}
\item $e$ is an integer called the exponent, 
\item $M$ is an integer called the integral significant.
\end{itemize}
The exponent satisfies $e_{min}\leq e\leq e_{max}$ while the integral significant 
satisfies $|M| \leq \beta^p - 1$. 

Let us compute the exponent and the integral significant of the number $x=0.1$.
The exponent is easily computed by the formula 
\begin{eqnarray}
e = \lfloor \log_2(|x|) \rfloor,
\end{eqnarray}
where the $\log_2$ function is the base-2 logarithm function.
In the case where an underflow or an overflow occurs, the value of 
$e$ is restricted into the minimum and maximum exponents range.
The following session shows that the binary exponent associated 
with the floating point number 0.1 is -4.
\begin{lstlisting}
-->format(25)
-->x = 0.1
 x  =
    0.1000000000000000055511  
-->e = floor(log2(x))
 e  =
  - 4.  
\end{lstlisting}
We can now compute the integral significant associated with this 
number, as in the following session.
\begin{lstlisting}
-->M = x/2^(e-p+1)
 M  =
    7205759403792794.  
\end{lstlisting}
Therefore, we deduce that the integral significant is equal to the decimal 
integer $M=7205759403792794$.
This number can be represented in binary form as the 53 binary digit number 
\begin{eqnarray}
M = 11001100110011001100110011001100110011001100110011010.
\end{eqnarray}
We see that a pattern, made of pairs of 11 and 00 appears. 
Indeed, the real value 0.1 is approximated by the following infinite 
binary decomposition:
\begin{eqnarray}
0.1 = \left(\frac{1}{2^0} + \frac{1}{2^1} + \frac{0}{2^2} + \frac{0}{2^3} + \frac{1}{2^4} + \frac{1}{2^5} + \ldots \right) \cdot 2^{-4}.
\end{eqnarray}

We see that the decimal representation of $x=0.1$ is made of a finite number of digits while the 
binary floating point representation is made of an infinite sequence of digits.
But the double precision floating point format must represent this number 
with 53 bits only. 

Notice that, the first digit is not stored in the binary double format, since it is 
assumed that the number is \emph{normalized} (that is, the first digit is assumed 
to be one). Hence, the leading binary digit is \emph{implicit}.
This is why there is only 52 bits in the mantissa, while we use 53 bits for the precision $p$.
For the sake of simplicity, we do not consider denormalized numbers in this discussion. 

In order to analyze how the rounding works, we look more carefully to the 
integer $M$, as in the following experiments.
\begin{lstlisting}
-->7205759403792793 * 2^(-4-53+1)
 ans  =
    0.0999999999999999916733  
 
-->7205759403792794 * 2^(-4-53+1)
 ans  =
    0.1000000000000000055511  
\end{lstlisting}
We see that the real number 0.1 is between two consecutive floating point 
numbers:
\begin{eqnarray}
7205759403792793 \cdot 2^{-4-53+1} < 0.1 < 7205759403792794 \cdot 2^{-4-53+1}.
\end{eqnarray}
There are four rounding modes in the IEEE floating point standard. The 
default rounding mode is \emph{round to nearest}, which rounds to the nearest floating 
point number. In case of a tie, the rounding is performed to the only one of these 
two consecutive floating point numbers whose integral significant is even.
In our case, the distance from the real $x$ to the two floating point 
numbers is 
\begin{eqnarray}
|0.1 - 7205759403792793\cdot 2^{-4-53+1}| &=& 8.33\cdots 10^{-18},\\
|0.1 - 7205759403792794\cdot 2^{-4-53+1}| &=& 5.55\cdots 10^{-18}.
\end{eqnarray}
(The previous computation is performed with a symbolic computation system, not with 
Scilab). 
Therefore, the nearest is the second integral significant. This is why the integral 
significant $M$ associated with $x=0.1$ is equal to $7205759403792794$, which leads to 
$fl(0.1)\approx 0.100000000000000005$.

On the other side, $x=0.9$ is also not representable 
as an exact binary floating point number (but 1.0 is exactly represented). 
The floating point binary representation of $x=0.9$ is associated with 
the exponent $e=-1$ and an integral significant between 8106479329266892 and 8106479329266893.
The integral significant which is nearest to $x=0.9$ is 8106479329266893, which is associated with 
the approximated decimal number $fl(0.9)\approx 0.90000000000000002$. 

Then, when we perform the subtraction "1.0-0.9", the decimal representation of the 
result is $fl(1.0)-fl(0.9)\approx 0.09999999999999997$, which is different from 
$fl(0.1)\approx 0.100000000000000005$.

\subsection{Why $sin(\pi)$ is rounded}

In this section, we present a brief explanation of the following 
Scilab 5.1 session, where the function sinus is applied to the 
number $\pi$.

\begin{lstlisting}
-->format(10)
 ans  =
    0.  
-->sin(%pi)
 ans  =
    1.225D-16  
\end{lstlisting}

This article is too short to make a complete presentation 
of the computation of elementary functions. The interested 
reader may consider the direct analysis of the Fdlibm library
as very instructive \cite{fdlibm}.
Muller presents in "Elementary Functions" \cite{261217}
a complete discussion on this subject.

In Scilab, the $\sin$ function is connected to a 
fortran source code (located in the \emph{sci\_f\_sin.f} file), where 
we find the following algorithm:
\begin{lstlisting}
do i = 0 , mn - 1
  y(i) = sin(x(i))
enddo
\end{lstlisting}
The \scivar{mn} variable contains the number of elements in the matrix,
which is stored as the raw array \scivar{x}. This implies that 
no additionnal algorithm is performed directly by Scilab and 
the $\sin$ function is computed by the mathematical library provided 
by the compiler, i.e. by gcc under Linux and by Intel's Visual Fortran under Windows.

Let us now analyze the algorithm which is performed by the mathematical library 
providing the $\sin$ function. 
In general, the main structure of these algorithms is the following:
\begin{itemize}
\item scale the input $x$ so that in lies in a restricted
interval, 
\item use a polynomial approximation of the local 
behavior of $\sin$ in the neighborhood of 0.
\end{itemize}

In the Fdlibm library for example, the scaling interval is 
$[-\pi/4,\pi/4]$. 
The polynomial approximation of the $\sin$ function has the general form
\begin{eqnarray}
sin(x) &\approx& x + a_3x^3 + \ldots + a_{2n+1} x^{2n+1}\\
&\approx & x + x^3 p(x^2)
\end{eqnarray}
In the Fdlibm library, 6 terms are used.

For the \scifun{atan} function, which is 
used to compute an approximated value of $\pi$, the process is the same.
This leads to a rounding error in the representation of $\pi$ which is 
computed by Scilab as $4*atan(1.0)$.
All these operations are guaranteed with some precision, when applied to a 
number in the scaled interval. For inputs outside the scaling interval, the accuracy
depends on the algorithm used for the scaling.

All in all, the sources of errors in the floating point computation of 
$\sin(\pi)$ are the following
\begin{itemize}
\item the error of representation of $\pi$,
\item the error in the scaling,
\item the error in the polynomial representation of the function $\sin$.
\end{itemize}
Since the value of $\sin(\pi)$ is close to the machine epsilon corresponding to 
IEEE double precision floating point numbers (i.e. close to $\epsilon \approx 2.220 \cdot 10^{-16}$),
we see that the result is the best possible.

\subsection{One more step}

In fact, it is possible to reduce the number of 
significant digits of the sine function to as low as 0 significant digits.
We mathematical have $sin(2^n \pi) = 0$, but this can be very inaccurate with 
floating point numbers. In the following Scilab session, we compute $\sin(2^{10i} \pi)$
for $i=1$ to 5.
\begin{lstlisting}
-->sin(2.^(10*(1:5)).*%pi).'
 ans  =
  - 0.0000000000001254038322  
  - 0.0000000001284092832066  
  - 0.0000001314911060035225  
  - 0.0001346468921407542141  
  - 0.1374419777062635961151  
\end{lstlisting}

For $sin(2^{50}\pi)$, the result is very far from being zero. This computation
may sound \emph{extreme}, but it must be noticed that it is inside the 
IEEE double precision range of values, since $2^{50} \approx 3.10^{15} \ll 10^{308}$.
If accurate computations of the $\sin$ function are required for large values of 
$x$ (which is rare in practice), the solution may be to use multiple precision 
floating point numbers, such as in the MPFR library \cite{MPFRWeb,Fousse:2007:MMP}, based on the 
Gnu Multiple Precision library \cite{GMPWeb}. 

%If you know a better algorithm, based on double precision only, 
%which allows to compute accurately such kind of values, the Scilab 
%team will surely be interested to hear from you !



%\input{pythagoreansum}

%% Bibliography


\addcontentsline{toc}{section}{Bibliography}
\bibliographystyle{plain}
\bibliography{scilabisnotnaive}

\addcontentsline{toc}{section}{Index}
\printindex

\end{document}

