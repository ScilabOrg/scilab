% Copyright (C) 2008-2010 - Consortium Scilab - Digiteo - Michael Baudin
%
% This file must be used under the terms of the 
% Creative Commons Attribution-ShareAlike 3.0 Unported License :
% http://creativecommons.org/licenses/by-sa/3.0/

\chapter{Conclusion}

That tool might be extended in future releases so that it provides the following features :
\begin{itemize}
\item Kelley restart based on simplex gradient [9],
\item C-based implementation (a prototype is provided in appendix B),
\item parallel implementation of the DIRECT algorithm,
\item implementation of the Hook-Jeeves and Multidimensional Search methods [9]
\item parallel implementation of the Nelder-Mead algorithm. See for example [21]. 
?This paper generalizes the widely used Nelder and Mead (Comput J 
7:308?313, 1965) simplex algorithm to parallel processors. Unlike most 
previous parallelization methods, which are based on parallelizing the 
tasks required to compute a specific objective function given a vector 
of parameters, our parallel simplex algorithm uses parallelization at 
the parameter level. Our parallel simplex algorithm assigns to each 
processor a separate vector of parameters corresponding to a point on a 
simplex. The processors then conduct the simplex search steps for an 
improved point, communicate the results, and a new simplex is formed. 
The advantage of this method is that our algorithm is generic and can be 
applied, without re-writing computer code, to any optimization problem 
which the non-parallel Nelder?Mead is applicable. The method is also 
easily scalable to any degree of parallelization up to the number of 
parameters. In a series of Monte Carlo experiments, we show that this 
parallel simplex method yields computational savings in some experiments 
up to three times the number of processors.?
\end{itemize}

